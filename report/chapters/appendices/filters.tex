\chapter{Filters}

\begin{enumerate}
	\item FIR - Finite impulse response filter
	\item IRR - Infinite impulse response filter
\end{enumerate}

\begin{enumerate}
	\item Low pass filter
	\item High pass filter
	\item Band filter
\end{enumerate}

\subsection{Audio Effects}

\begin{enumerate}
	\item Reverb
	\item Echo
	\item Distortion
	\item Chorus
\end{enumerate}

\paragraph{Goal}
- To make a pipelined MIMD with cores executing different transforms and
filters.
	E.g. Input $=>$ FFT $=>$ IIR $=>$ IFFT $=>$ Output

\paragraph{Buffering}
- The cores in the pipeline should have a shared memory in between each core to
communicate
	E.g. Mem $<=>$ Core $<=>$ Mem $<=>$ Core ... $<=>$ Mem

\begin{itemize}
	\item Insight A: FFT transforms a window of samples into a Frequency image
for this window. The window is created by N samples.
	\item Effect: Data will be transfered through the pipeline in big chunks. One
new chunk will be transfered on each ADC cycle.
	\item Proposed solution: Each shared memory should be divided in two equally
sized pieces, A and B. $Core_i$ uses $A_{i-1}$ as input and $B_{i+1}$ as
output on cycle $2j$ and $B_{i-1}$ as input and $A_{i+1}$ on cycle $2j+1$ for
$j = 0,1,2,3,4$
\end{itemize}

\begin{itemize}
	\item Insight $B_1$: If the a FFT is not applied a N
	\todo{N is already used for the number of samples per transform} bit sample
	will arrive at each ADC cycle.
	\item Insight $B_2$: A filter needs historic samples.
	\item Effect: Data will be transfered in small quantities, if the previously
scheme is used for samples at the start of the buffer historic data will not be
available.
	\item Proposed solution: Converge the buffer design to a circular buffer
design either at sample level or at some division of the buffer greater than 2.
\end{itemize}
