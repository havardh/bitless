% !TEX root = ../report.tex

\chapter{Conclusion}

This chapter contains our conclusion, as well as our opinion of the project,
pitfalls we believe later generations can avoid by keeping in mind from early on
in the project, and finally our budget.

\clearpage
\section{Project conclusion}
\todo[inline]{Repeat the main results of our work (repeat and summarize results from result chapter)}

\clearpage
\section{Potential Improvements}

\todo[inline]{The way forward}

reduce pcb cost by only using drill holes

\clearpage
\section{Our opinion of the project}

This project has been immensely demanding, requiring a lot of hard work and many
late hours. Even so, we, the 10 persons the group is comprised of found it to be
a very interesting and rewarding project. From the start of the course, the
group was split up into three subgroups, FPGA, software, and PCB. This division
made it simple to divide the labour between those three respective groups. For
all three of the subgroups, many many manhours were spent on just learning our
respective fields. While previous reports were made available to us, 40+ pages
is too much for any new group which have never programmed a Silicon Labs
microcontroller, used Altium design a PCB, or created processor design in VHDL
before.

On behalf of the FPGA subgroup in this project, we strongly feel that VHDL needs
to have been learned before attempting this project. That we had the course
TDT4255\cite{tdt4255} simultaneously with this course was both a huge boon and a
painful curse. An example of why this is so, is that our attempts at Timing
Simulations (insufficiently explained in the compendium\cite{tdt4255-compendium}
of TDT4255), were not started until there were less than 4 days left of this
project. TDT4255 is a great course for learning VHDL and how it is used, but our
opinion is that it should be held an earlier semester.

If those responsible for the course, or the reports of previous groups, could
compile a summarized list of ``how to avoid these pitfalls'' (especially those
that every new generation fall into), that would be of immense help throughout
the project. Both the design phase at the beginning could be completed with more
self-assurance, and less hassle and hacks at the end when the deadline is up for
having completed a working demo.

\subsection{Pitfalls to be avoided}\label{conclusion:pitfalls}

Below we have compiled a list of issues we believe new groups should keep in
mind when starting out on the TDT4295 adventure:

\todo[inline]{Fyll inn listen under for senere generasjoner sin skyld.}
\begin{itemize}
	\item FPGA
	\begin{itemize}
		\item
	\end{itemize}
	\item Software
	\begin{itemize}
		\item
	\end{itemize}
	\item PCB
	\begin{itemize}
		\item
	\end{itemize}
\end{itemize}

\clearpage
\section{Expenses}
The project expenses are listed in Table~\ref{tab:budget}. The budget was exceeded by 884 NOK.

\begin{table}[H]
	\centering
	\begin{tabular}{|l|l|}
		\hline
		\textbf{Purchase} & \textbf{Cost} \\
		\hline
		\hline
		FPGA and microcontroller & 1.687 NOK\\
		\hline
		Components & 2.069 NOK\\
		\hline
		10 custom PCBs & 7.128 NOK \\
		\hline
		\hline
		\textbf{\textit{Sum:}} & \textit{10.884 NOK}\\
		\hline
	\end{tabular}
	\caption{Budget for the Bitless project}
	\label{tab:budget}
\end{table}
