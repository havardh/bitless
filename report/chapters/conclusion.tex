% !TEX root = ../report.tex

\chapter{Conclusion}

\todo[inline]{Repeat the main results of our work (repeat and summarize results from result chapter)}

\section{Potential Improvements}

\todo[inline]{The way forward}

reduce pcb cost by only using drill holes

\section{Our opinion of the project}
This project has been immensely demanding, requiring a lot of hard work and many late hours.
Even so, we, the 10 persons the group is comprised of found it to be a very interesting and rewarding project.
From the start of the course, the group was split up into three subgroups, FPGA, software, and PCB.
This division made it simple to divide the labour between those three respective groups.
For all three of the subgroups, many many manhours were spent on just learning our respective fields.
While previous reports were made available to us, 40+ pages is too much for any new group which have never programmed a Silicon Labs microcontroller, used Altium design a PCB, or created processor design in VHDL before.
If those responsible for the course, or the reports of previous groups, could compile a summarized list of ``how to avoid these pitfalls'' (especially those that every new generation fall into), that would be of immense help throughout the project.
Both the design phase at the beginning could be completed with more self-assurance, and less hassle and hacks at the end when the deadline is up for having completed a working demo.


\subsection{Pitfalls to be avoided}\label{conclusion:pitfalls}
Below we have compiled a list of issues we believe new groups should keep in mind when starting out on the TDT4295 adventure:
\todo[inline]{Fyll inn listen under for senere generasjoner sin skyld.}
\begin{itemize}
	\item FPGA
	\begin{itemize}
		\item Learn VHDL a semester earlier. The subject TDT4255 should have been a subject students could take earlier.
But it should most definitively not be held at the same time as TDT4295.
		\item Timing simulations to ensure the correct working of design needs to be explained better.
Our only resource on how to do this is the TDT4255 compendium\cite{tdt4255-compendium}, which explains this with different software than what we currently have access to at the lab.
	\end{itemize}
	\item Software
	\begin{itemize}
		\item
	\end{itemize}
	\item PCB
	\begin{itemize}
		\item
	\end{itemize}
\end{itemize}

\section{Expenses}
The project expenses are listed in Table~\ref{tab:budget}. The budget was exceeded by 884 NOK.

\begin{table}[H]
	\centering
	\begin{tabular}{|l|l|}
		\hline
		\textbf{Purchase} & \textbf{Cost} \\
		\hline
		\hline
		FPGA and microcontroller & 1.687 NOK\\
		\hline
		Components & 2.069 NOK\\
		\hline
		10 custom PCBs & 7.128 NOK \\
		\hline
		\hline
		\textbf{\textit{Sum:}} & \textit{10.884 NOK}\\
		\hline
	\end{tabular}
	\caption{Budget for the Bitless project}
	\label{tab:budget}
\end{table}
