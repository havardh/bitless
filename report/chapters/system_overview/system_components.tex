% !TEX root = ../../report.tex

\section{System Components}

\subsection{MCU} The microcontroller unit is a Giant Gecko by Energy Micro
(EFM32GG990F1024) which is an energy friendly MCU capable of shutting down
components that are not in use. It does this by defining several levels of sleep
mode where the outer stage has everything on, and for each step inwards a new
set of components are shut off.\missingfigure{add the fancy energy circle
figure?} This enables the MCU to keep I/O components working while shutting off
the processor unit on the chip.\todo{review, rewrite}

The MCU handles all I/O, effectively acting as the link between the FPGA and the
peripherals. Sound input is either gathered from a file on the SD-card or as a
stream from the audio-in jack on the board, and then passed on to the FPGA. In
addition it controls the running program on the FPGA, providing an easy channel
for changing FPGA behavior.\todo{review, rewrite}

\subsection{FPGA} The FPGA is programmed with a MIMD architecture in mind to
enable task parallelism. While this is great for performance, it comes at the
cost of a more complex process architecture, which in turn increases the size of
each core. This limits the number of cores available on the FPGA, but we were
able to fit ten\todo{update with final number} cores which is enough to apply
the filters we want.\todo{review, rewrite}

\subsection{Memory} The MCU has an internal memory of 128KByte, which is not
enough to work comfortably with audio files loaded from the SD-card, so a
decision was made to add more memory on the board. In addition, if something
were to go wrong with the audio-in channel and SD-card, having extra memory can
work as a third option for storing input.\todo{review, rewrite}

\subsection{I/O} Adding input and output interfaces makes handling data easier
as we do not have to go though the debug interface to insert and fetch data from
the chip. Thus, we decided to add the following components to the
board.\todo{review, rewrite}

\subsubsection{Micro USB} USB \todo{review, rewrite}

\subsubsection{Micro SD} SD-card \todo{review, rewrite}

\subsubsection{Minijacks} We added two jack channels, one for input and one for
output, as taking input from a recording og playback device is much easier if
you can just plug it in rather than having to transfer data through a debugging
interface, and the same goes for output.\todo{review, rewrite}

\subsubsection{Buttons and LEDs} What's there to say.. They're
awesome.\todo{review, rewrite}
