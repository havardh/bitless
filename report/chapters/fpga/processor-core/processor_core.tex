\FloatBarrier
\section{Processor Core}\label{section:fpga-processor-core}
\todo[inline]{How is the processor designed and \emph{why!}}

\FloatBarrier
\subsection{Considerations}
\todo[inline]{Describe the pipelines of the cpu, and why it is implemented/
designed the way it is.}

The processor core was implemented with focus on three aspects:

\begin{enumerate}
	\item Real-time sound manipulation
	\item A focus on energy-efficiency.
	\begin{enumerate}
		\item Since we are implementing this on an FPGA, the focus will be on
simulating (through having signals in VHDL simulating energy-efficiency switch
choices) opportunities and the potential for energy-efficiency. This because
it will be most impossible to program and test an FPGA chip not meant
specifically for energy-efficiency purposes to be tested with respect to
energy-efficiency.
	\end{enumerate}
	\item An architectural design permitting the manipulation of the most common
sound-effects on the above real-time datastream, and yet still having such a
generalized design so as to permit the running of most standard MIPS
instructions.
\end{enumerate}

With these three points in mind, the decision was made to base the design on a
processor very similar to the generalized pipelined MIPS processor that the
subject TDT4255\cite{tdt4255} has its students implementing in the course
exercises.

\FloatBarrier
\FloatBarrier
\subsection{ALU}\label{subsec:fpga-alu}

Which operations were implemented in the ALU was decided based on behalf of what
was neccessary for the Fourier-transform, sound processing, and on whether the
instruction set had room for the encoding of the operations.

Division was not implemented in hardware in order for all instructions to be
single-cycle. However, division can be done using floating point multiplication
or by a regular long division algorithm in software.

The ALU implementation receives two 16-bit registers as inputs, and outputs a
32-bit register. However, only the lower 16-bits are currently used. Previously,
the intention was for multiplication operations to output all 32-bits.

\subsubsection{Internal memory}

Since the instruction words can only address two registers at the time, the ALU
unit can store the values previously loaded from constant memory internally.
``Multiply and accumulate'' is the longest critical path, and which is used
during the SD-FT \ref{appendix:sd-ft}. This instruction needs the constant value
previously loaded into the ALU with the ``Load from constant memory''
instruction. Hence why the ALU can store a constant memory value internally.


\FloatBarrier
\subsection{Floating-point implementation design choices}

\FloatBarrier
\subsection{Memory access}
\todo[inline]{Why constant values go to the alu, and how it changes pipeline
layout. How the core accesses memory.}