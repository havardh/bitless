\FloatBarrier

\section{Design Planning}\label{sec:fpga-design}
\todo{I think all this can be moved to the introduction and the FFT stuff here
can be merged with the FFT stuff there - KKS}

The processor was designed top-down for sound processing. As sound processing
often requires fourier transforms, several solutions were considered for how
to implement hardware support for this.

Designing a core specifically for doing the fourier transform and its inverse
was considered, but could cause increased complexity when desiging the
audio processing pipelines. A variant of this idea was to equip the first
and the last core of each audio pipeline with instructions accelerating
the most compute intensive parts of the fourier transforms, but this
would cause the processors to not be homogenous.

In the end, implementing a simple floating point unit in each of the cores
would not only allow acceleration of the fourier transforms, but also manipulation
of the resulting samples in the frequency domain.

