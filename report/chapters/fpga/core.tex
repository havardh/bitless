\FloatBarrier
\section{Processor Core}\label{section:fpga-processor-core}
\todo[inline]{How is the processor designed and \emph{why!}}

\FloatBarrier
\subsection{Considerations}

The processor core was implemented with focus on three aspects:

\begin{enumerate}
	\item Real-time sound manipulation
	\item A focus on energy-efficiency.
	\begin{enumerate}
		\item Since we are implementing this on an FPGA, the focus will be on
simulating (through having signals in VHDL simulating energy-efficiency switch
choices) opportunities and the potential for energy-efficiency. This because
it will be most impossible to program and test an FPGA chip not meant
specifically for energy-efficiency purposes to be tested with respect to
energy-efficiency.
	\end{enumerate}
	\item An architectural design permitting the manipulation of the most common
sound-effects on the above real-time datastream, and yet still having such a
generalized design so as to permit the running of most standard MIPS
instructions.
\end{enumerate}

With these three points in mind, the decision was made to base the design on a
processor very similar to the generalized pipelined MIPS processor that the
subject TDT4255\cite{tdt4255} has its students implementing in the course
exercises.

\FloatBarrier
\subsection{Floating-point implementation design choices}

\FloatBarrier
\subsection{Memory access}
\todo[inline]{Why constant values go to the alu, and how it changes pipeline
layout. How the core accesses memory.}
