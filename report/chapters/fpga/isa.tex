% !TEX root = ../../report.tex
\section{Instruction Set Architecture}\label{section:fpga-isa}

The instruction set of the processor cores where designed before work
started on the processor cores themselves. This was done in order
to build the processor core based on which operations were actually
needed. This resulted in two distinct drafts of the instruction set
architecture.

\subsection{First Draft}
The first draft of the ISA was devised early in the design process.
It was decided to use a processor architecture with 16-bit instruction words
and 32 general purpose registers. The instructions are divided in three
different classes: load/store instructions, arithmetic instructions, and
miscellaneous instructions.

All instructions have a common group field in the two most significant bits,
which is used to select between one of four instruction groups, as shown in
table \ref{tab:group_fmt}.

For our ALU operations, it was decided to use four condition flags, listed in
table \ref{tab:condition_flags}. These are stored in a dedicated status and
control register.

\begin{table}[ht]
    \centering
    \begin{tabular}{|l l|}
        \hline
        \textbf{Group field} & \textbf{Meaning of field} \\
        \hline
        \texttt{0b00} & Load/store instructions \\
        \hline
        \texttt{0b01} & \multirow{2}{*}{Arithmetic instructions} \\
        \texttt{0b10} & \\
        \hline
        \texttt{0b11} & Miscellaneous instructions \\
        \hline
    \end{tabular}
    
    \caption{Format of the group field}
    \label{tab:group_fmt}
\end{table}

\begin{table}[h]
	\centering
	\begin{tabular}{|l l|}
		\hline
		Z & Zero flag \\
		N & Negative flag \\
		C & Carry flag \\
		V & Overflow flag \\
		\hline
	\end{tabular}
	\caption{Condition flags}
	\label{tab:condition_flags}
\end{table}


\subsubsection{Load/store instructions}
The format of our load/store instructions is illustrated in figure
\ref{fig:ls_instr_format}, and the instructions considered for this format
are listed in table \ref{tab:ls_instrs}.

\begin{figure}[h]
    \centering
    \begin{bytefield}[endianness=big,bitwidth=0.05\linewidth]{16}
        \bitheader{0-15} \\
        \bitbox{2}{Group} &
        \bitbox{2}{Funct} &
        \bitbox{12}{Immediate}
    \end{bytefield}

    \caption{Format of load/store instructions}
    \label{fig:ls_instr_format}
\end{figure}

\begin{table}[h]
    \centering
    \begin{tabular}{|l l l|}
        \hline
        \textbf{Funct} & \textbf{Mnemonic} & \textbf{Instruction} \\
        \texttt{0b00} & \textsc{ldm} & Load immediate \\
        \texttt{0b10} & \textsc{ldi} & Load from input buffer \\
        \texttt{0b10} & \textsc{ldo} & Load from output buffer \\
        \texttt{0b11} & \textsc{sto} & Store to output buffer \\
        \hline
    \end{tabular}

    \caption{Load/store instructions}
    \label{tab:ls_instrs}
\end{table}

\subsubsection{Arithmetic instructions}
The format of the arithmetic instructions is illustrated in figure
\ref{fig:arith_instr_format}.
Implemented instructions are listed in table \ref{tab:arith_instrs}. In order
to support other variants of the same basic instructions, such as add-with-carry
or multiply-and-accumulate. The operations are implemented such that the result
of an operation is written back to the first operand register, $ra = ra~op~rb$.

\begin{figure}[h]
	\centering
	\begin{bytefield}[endianness=big,bitwidth=0.05\linewidth]{16}
		\bitheader{0-15} \\
		\bitbox{2}{Group} &
		\bitbox{2}{Funct} &
		\bitbox{2}{Option} &
		\bitbox{5}{Reg A} &
		\bitbox{5}{Reg B} 
	\end{bytefield}

	\caption{Arithmetic instructions}
	\label{fig:arith_instr_format}
\end{figure}

\begin{table}[ht]
    \centering
    \begin{tabular}{|l l l l|}
        \hline
        \textbf{Group} & \textbf{Funct} & \textbf{Mnemonic} & \textbf{Instruction} \\
        \hline
        \multirow{4}{*}{\textsc{0b01}}
            & \texttt{0b00} & \textsc{add} & Add\\
            & \texttt{0b01} & \textsc{sub/cmp} & Subtract/compare\\
            & \texttt{0b10} & \textsc{mul} & Multiply\\
            & \texttt{0b11} & \textsc{mov} & Move between registers\\
        \hline
        \multirow{4}{*}{\textsc{0b10}} 
            & \texttt{0b00} & \textsc{and} & Bitwise and\\
            & \texttt{0b01} & \textsc{or} & Bitwise or\\
            & \texttt{0b10} & \textsc{xor} & Bitwise exclusive or\\
            & \texttt{0b11} & \textsc{not} & Bitwise invert\\
        \hline
    \end{tabular}

    \caption{Implemented arithmetic instructions}
    \label{tab:arith_instrs}
\end{table}


\subsubsection{Miscellaneous instructions}
The format of the miscellaneous instructions is illustrated in figure
\ref{fig:misc_instr_format}. The implemented instructions are listed in table
\ref{tab:misc_instrs}.

\begin{figure}[h]
    \centering
    \begin{bytefield}[endianness=big,bitwidth=0.05\linewidth]{16}
        \bitheader{0-15} \\
        \bitbox{2}{Group} &
        \bitbox{1}{F} &
        \bitbox{4}{Param} &
        \bitbox{9}{Immediate}
    \end{bytefield}

    \caption{Format of the miscellaneous instructions}
    \label{fig:misc_instr_format}
\end{figure}

\begin{table}[h]
    \centering
    \begin{tabular}{|l l l|}
        \hline
        \textbf{F field} & \textbf{Mnemonic} & \textbf{Instruction} \\
        \texttt{0} & \textsc{bcc} & Branch (with condition code) \\
        \texttt{1} & \textsc{shl/shr} & Shift left/Shift right\\
        \hline
    \end{tabular}

    \caption{Implemented miscellaneous instructions}
    \label{tab:misc_instrs}
\end{table}


\FloatBarrier
\subsection{Second Draft}

Due to the requirements of doing an FFT on the processor, floating point
instructions for addition, multiplication and subtraction were added.
In addition, most instructions were moved into the group previously called
arithmetic operations. Some new data movement instructions were added, in part
because of the addition of a separate memory area for storing constants.

\subsection{Register-based instructions}

The group previously named arithmetic operations now contains most of the
implemented instructions, including load/store instructions. They all have
in common tat they have at least one register as operand. The format of the
instructions are illustrated in figure \ref{fig:regbased_instrs_format}. The
implemented functions can be found in table \ref{tab:regbased_instrs}.

\begin{figure}[h]
	\centering
	\begin{bytefield}[bitwidth=0.05\linewidth]{16}
		\bitheader{0-15}	\\
		\bitbox{2}{Group}	&
		\bitbox{2}{Funct}	&
		\bitbox{2}{Opt}		&
		\bitbox{5}{Reg A}	&
		\bitbox{5}{Reg B/Imm}
	\end{bytefield}

	\caption{Register-based instruction format}
	\label{fig:regbased_instrs_format}
\end{figure}

\begin{table}[H]
	\centering
	\begin{tabular}{|l l l l l|}
		\hline
		\textbf{Funct} & \textbf{Opt}  & \textbf{Mnemonic} & \textbf{Instruction} & \textbf{Operation} \\
	\hline
	\multicolumn{5}{|c|}{Group \texttt{0b00}} \\
	\hline
	\multirow{3}{*}{\texttt{0b00}}
		& \texttt{0b00} & \texttt{add \$ra, \$rb}  & Add registers & $\$ra \leftarrow \$ra + \$rb$ \\
		& \texttt{0b01} & \texttt{addi \$ra, imm} & Add immediate & $\$ra \leftarrow \$ra + imm$ \\
		& \texttt{0b10} & \texttt{fadd \$ra, \$rb} & Add registers (FP) & $\$ra \leftarrow \$ra + \$rb$ \\
	\multirow{4}{*}{\texttt{0b01}}
		& \texttt{0b00} & \texttt{sub \$ra, \$rb}  & Subtract registers & $\$ra \leftarrow \$ra - \$rb$ \\
		& \texttt{0b01} & \texttt{fsub \$ra, \$rb} & Subtract registers (FP) & $\$ra \leftarrow \$ra - \$rb$ \\
		& \texttt{0b10} & \texttt{cmp \$ra, \$rb}  & Compare & $cnd \leftarrow cnd(\$ra - \$rb)$ \\
		& \texttt{0b11} & - & compare fp / subtract imm? & - \\
	\multirow{4}{*}{\texttt{0b10}}
		& \texttt{0b00} & \texttt{mul \$ra, \$rb}  & Multiply registers & $\$ra \leftarrow \$ra * \$rb$ \\
		& \texttt{0b01} & \texttt{fmul \$ra, \$rb} & Multiply registers (FP) & $\$ra \leftarrow \$ra * \$rb$ \\
		& \texttt{0b10} & \texttt{fmla \$ra, \$rb} & Multiply-and-accumulate (FP) & $\$ra \leftarrow \$ra + \$rb * \$rc$ \\
		& \texttt{0b11} & \texttt{fmls \$ra, \$rb} & Multiply-and-subtract (FP) & $\$ra \leftarrow \$ra - \$rb * \$rc$ \\
	\multirow{2}{*}{\texttt{0b11}}
		& \texttt{0b00} & \texttt{shl \$ra, imm} & Shift left & $\$ra \leftarrow \$ra << imm$ \\
		& \texttt{0b01} & \texttt{shr \$ra, imm} & Shift right & $\$ra \leftarrow \$ra >> imm$\\
	\hline
	\multicolumn{5}{|c|}{Group \texttt{0b01}} \\
	\hline
	\multirow{2}{*}{\texttt{0b00}}
		& \texttt{0b00} & \texttt{and \$ra, \$rb} & And & $\$ra \leftarrow \$ra \wedge \$rb$ \\
		& \texttt{0b01} & \texttt{nand \$ra, \$rb} & Nand & $\$ra \leftarrow \neg(\$ra \wedge \$rb)$ \\
	\multirow{3}{*}{\texttt{0b01}}
		& \texttt{0b00} & \texttt{or \$ra, \$rb} & Or & $\$ra \leftarrow \$ra \vee \$rb$ \\
		& \texttt{0b01} & \texttt{nor \$ra, \$rb} & Nor & $\$ra \leftarrow \neg(\$ra \vee \$rb)$\\
		& \texttt{0b10} & \texttt{xor \$ra, \$rb} & Xor & $\$ra \leftarrow \$ra \oplus \$rb$\\
	\multirow{4}{*}{\texttt{0b10}}
		& \texttt{0b00} & \texttt{mov \$ra, \$rb} & Move & $\$ra \leftarrow \$rb$\\
		& \texttt{0b01} & \texttt{mvn \$ra, \$rb} & Move negative & $\$ra \leftarrow \neg\$rb$ \\
		& \texttt{0b10} & \texttt{i2f \$ra, \$rb} & Typecast (Int to FP) & $\$ra \leftarrow fp(\$rb)$ \\
		& \texttt{0b11} & \texttt{f2i \$ra, \$rb} & Typecast (Fp to int) & $\$ra \leftarrow int(\$rb)$ \\
	\multirow{4}{*}{\texttt{0b11}}
		& \texttt{0b00} & \texttt{lda \$ra, [\$rb]} & Load from input & $\$ra \leftarrow [\$rb]$ \\
		& \texttt{0b01} & \texttt{ldb \$ra, [\$rb]} & Load from output & $\$ra \leftarrow [\$rb]$ \\
		& \texttt{0b10} & \texttt{ldc \$ra, [\$rb]} & Load from constant buffer & $\$ra \leftarrow [\$rb]$ \\
		& \texttt{0b11} & \texttt{stb \$ra, [\$rb]} & Store to output & $[\$rb] \leftarrow \$ra$ \\
	\hline
	\end{tabular}
	\caption{Register-based instruction list}
	\label{tab:regbased_instrs}
\end{table}



\subsection{Load immediate instruction}
The load immediate instruction is used to load an immediate constant into a
register. The format of the instruction can be found in figure
\ref{fig:ldi_format}. The value is loaded into register \texttt{\$ri}.

\begin{figure}[h]
	\centering
	\begin{bytefield}[bitwidth=0.05\linewidth]{16}
		\bitheader{0-15} \\
		\bitbox{2}{Group} &
		\bitbox{14}{Imm}
	\end{bytefield}

	\caption{Load immediate format}
	\label{fig:ldi_format}
\end{figure}


\subsection{Branch instruction}
The branch instruction checks the condition flags and jumps accordingly. By
checking for various combinations of the condition flags, many different
conditions can be checked for. The format of the instruction is illustrated
in figure \ref{fig:new_branch_format}. List for the encoding of the flags field
is coming later.

\begin{figure}[h]
	\centering
	\begin{bytefield}[bitwidth=0.05\linewidth]{16}
		\bitheader{0-15} \\
		\bitbox{2}{Group} &
		\bitbox{4}{Flags} &
		\bitbox{11}{Target}
	\end{bytefield}

	\caption{Branch instruction format}
	\label{fig:new_branch_format}
\end{figure}



\subsubsection{Special registers}

Due to the limited space in instruction words, only two registers at most can be
specified in an instruction. Some instructions, such as the load immediate and
load constant instructions do not specify any registers, only an immediate
offset constant.

The list of defined special registers can be found in table \ref{tab:specregs}.
The register number for these registers have not yet been determined.

\begin{centering}[h]
	\begin{tabular}{|l p{10.5cm}|}
		\hline
		\textbf{Register name} & \textbf{Register purpose} \\
		\hline
		\texttt{r0} & Zero register, hard coded to always contain 0 \\
		\texttt{r1} & Immediate register, contains the result of an \textsc{ldi} instruction\\
		\texttt{rc} & Constant register, provides operand to \textsc{fmla} and \textsc{fmls} instructions. Lies within the ALU. \\
		\texttt{rc} & Constant register, provides operand to \textsc{fmla} and \textsc{fmls} instructions. Lies within the ALU. \\
		\hline
	\end{tabular}

	\label{tab:specregs}
	%\caption{List of special registers}
	
\end{centering}



