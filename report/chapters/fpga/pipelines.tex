\FloatBarrier
\section{Sequential processing-core pipeline}\label{section:sequential-pipeline}
\todo{Feel free to switch out the title for something better}

\FloatBarrier
\subsection{Top-down architectural view}\todo{Maybe find better subsection
title?}

\missingfigure{We should have a figure of the pipeline here, we got several
in google docs, use/pretty up one of those should suffice to give an
architectural overview}

In the above \todo{insert reference to above figure}figure\footnote{insert
reference to above figure.}, we illustrate how the processor connects all the
processing cores together on the FPGA.

Though the \todo{insert reference to the same above figure}figure\footnote{
insert reference to the same above figure.} above shows how we connected the
EBI-bus\cite{efm_ebi} to each processing core, during normal non-debugging
runtime the cores only ``communicate'' in the manner described below, and not
through the EBI-bus. The EBI-bus is used only as a way for inter-communication
between the cores when debugging.

The manner in which the cores ``communicate'' with each other is instead by the
two-stage ring-buffers between them. For each clockcycle belonging to the
pipeline of processor cores, they switch which one of the ones preceding they
read from, and which of the ones following they write to\footnote{As mentioned
in the footnote on the introduction to this chapter, this is how the ``Multiple
Data'' part of the ``MIMD'' definition is realized in our processor.}.

\FloatBarrier
\subsection{Start-up}\label{subsection:fpga-pipeline-startup}

\todo[inline]{This is where we explain how the MCU sets up the pipeline, and
how/what signals are set (and to what they are set) when the processor starts
up.}

\subsection{In action}

The input of the FPGA comes from the EBI-bus and enters one of the \todo{Insert
reference here?}input-memory buffers of the first core in each of the
``sequential-core-pipelines'', where the first core works on the sample, before
giving it over to the next core in said pipeline.
