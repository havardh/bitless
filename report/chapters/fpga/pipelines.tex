\FloatBarrier
\section{Sequential processing-core pipeline}\label{section:sequential-pipeline}
\todo{Feel free to switch out the title for something better}

\FloatBarrier
\subsection{Top-down architectural view}\todo{Maybe find better subsection
title?}

\missingfigure{We should have a figure of the pipeline here, we got several
in google docs, use/pretty up one of those should suffice to give an
architectural overview}

In the above \todo{insert reference to above figure}figure \textbf{XX}, we
illustrate how the processor connects all the processing cores together on the
FPGA.

Though the \todo{insert reference to the same above figure}figure \textbf{XX}
above shows how we connected the EBI-bus\cite{efm_ebi} to each processing core,
during normal non-debugging runtime the cores only ``communicate'' in the manner
described below, and not through the EBI-bus. The EBI-bus is used only as a way
for inter-communication between the cores when debugging.

The manner in which the cores ``communicate'' with each other is instead by the
two-stage ring-buffers between them. For each clockcycle belonging to the
pipeline of processor cores, they switch which one of the ones preceding they
read from, and which of the ones following they write to (as this is how the
``Multiple Data'' part of the ``MIMD'' definition is realized in our processor).

\FloatBarrier
\subsection{Pipeline control register}\label{subsection:fpga-pipeline-startup}

\todo{This is where we describe the pipeline control register.}

\subsection{Start-up}\label{subsection:fpga-pipeline-startup}

\todo[inline]{This is where we explain how the MCU sets up the pipeline, and
how/what signals are set (and to what they are set) when the processor starts
up.}

When starting up, the first thing that has to happen when attempting to
initialize the FPGA-chip, is to use the EBI-bus to write to reset bit (the 15th
(MSB)), should be used when initializing the FPGA in the toplevel
FPGA-control register (which is described in section
\ref{subsection:fpga-design-toplevel}).

This then resets \todo{Resets what in the FPGA??}.

After the reset has been accomplished, the reset bit in the toplevel
FPGA-control register needs to be cleared again. The following step will then be
to turn on the ``stopmode'' bit (the 7th bit) of the pipeline-control
register (as described in section \ref{subsection:fpga-pipeline-startup}),
to stop all of the processor cores on the FPGA.

While all the processor cores are off, the instruction memory of each core needs
to be filled with the instructions said core will run via the EBI-bus. Before
the stopmode bit is cleared and the cores are set to run, the first datasamples
should be prepared and in position with the help of the EBI bus into the input
buffers on each pipeline.

Then what remains is the final stage of turning the stopmode bit back on, and
the FPGA should be running as programmed.

\subsection{In action}

The input of the FPGA comes from the EBI-bus and enters one of the input-memory
buffers (as detailed in section \ref{subsection:fpga-internmem-IO}), of the
first core in each of the ``sequential-core-pipelines'', where the first core
works on the sample, before giving it over to the next core in said pipeline.
