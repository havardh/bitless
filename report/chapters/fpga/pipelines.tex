\FloatBarrier
\section{Sequential processing-core pipeline}\label{section:sequential-pipeline}\todo{Feel free to switch out the title for something better}

\FloatBarrier
\subsection{Top-down architectural view}\todo{Maybe find better subsection title?}

\missingfigure{We should have a figure of the pipeline here, we've got several in google docs, use/pretty up one of those should suffice to give an architectural overview}

In the above figure, \todo[inline]{insert reference to above figure} we
illustrate how the processor connects all the processing cores together on the
FPGA.

As you can see, we have each processor core connected to the \todo{Maybe insert reference?} EBI-bus, for debugging purposes.

The manner in which the cores ``communicate'' with each other is instead by the
two-stage ring-buffers between them. For each clockcycle belonging to the
pipeline of processor cores, they switch which one of the ones preceding they
read from, and which of the ones following they write to.

\todo{This would fit better as a footnote?}(This is where the ``Multiple Data'' part of the ``MIMD'' definition is enacted
in our processor.)

\FloatBarrier
\subsection{Start-up}

\todo[inline]{This is where we explain how the MCU sets up the pipeline}


\FloatBarrier
\subsection{In action}

So the input of the FPGA goes to the first core in each ``core-pipeline'', where
the first core works on the sample, before giving it over to the next core in
said pipeline.

\FloatBarrier
\subsection{Debugging6}


