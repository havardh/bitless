\chapter{Results}\label{chapter:results}

This chapter details the results from all the tests in the previous chapter. It
also details whether the system as a whole has satisfied the project assignment.

\clearpage
\section{Assignment results}

While \textit{ChaosM} was able to fully function with one pipeline in
simulation, there was insufficient time to successfully integrate it on the FPGA
chip on the system. Therefore, the actual audio processing was never
successfully tested on the system. Consequently many of the Functional
Requirement Tests (FRTs) were not tested. However, the I/O controller was
successfully tested against both the SRAM and the FPGA chip when configured as
RAM. The PCB was also thoroughly tested. Following are the testing results of
these two components.

%The below input is a subsection of the above section
\subsection{I/O controller Results}

The main objective for the I/O controller was to control all periherals in the system.
This was successfully accomplished. Although \textit{ChaosM} on the FPGA was not able
to produce any stable results, the I/O controller was able to communicate with the
FPGA though the EBI bus in a deterministic way. This was proved by setting the
FPGA up as RAM and writing and reading to it over EBI.

%The below input is a subsection of the above section
% !TEX root = ../../report.tex

\subsection{PCB Results (fixes)}

Due to the manual work involved in designing the PCB, and because it was fairly
unfamiliar ground for everyone involved, there were many things that could go
wrong. This section describes everything that was wrong with the PCB along with
what was done to fix it.

The first thing to happen was smoke appearing from the linear regulator upon
plugging in the power source. As it turned out, the footprint was mismatched
with the pinout of the component. The footprint had been fixed early on in the
process, but unfortunately the schematics had not been updated. The same turned
out to have happened with the low frequency crystal connected to the MCU,
although this component did not produce any smoke.

Both of the above problems where solved by extra wiring. For the crystal, where
pin 1 and 4 was suppose to be used, pin 1 and 2 were used instead. As pins 2 and
3 were unconnected, a patch cable was soldered between pin 4 and 2.

\begin{figure}[H]
    \centering
    \includegraphics[scale=0.1]{figures/results/pcb/lfxtal}
    \caption{Low frequency crystal hack.}
    \label{fig:res:lfxtal}
\end{figure}



The linear regulator was slightly more advanced as pin 1 and 5, and 2, 3 and 4,
had been swapped. The solution was to solder six short patch cables on the pads
themselves, and then match the feet of the linear regulator with the right
cables. This resulted in the linear regulator being mounted about 1 cm above the
card.

\begin{figure}[H]
    \centering
    \includegraphics[scale=0.1]{figures/results/pcb/linear-regulator}
    \caption{Linear voltage regulator hack.}
    \label{fig:res:linreg}
\end{figure}



In addition, due to a typo in the schematic, the data lines of the EBI bus were
only connected to the MCU and external RAM, not to the FPGA. Again, this was
solved by patch cables between the external RAM's data pins and all available
FPGA GPIO pins. The result was a fully functional bus as was originally
intended, at the expense of all other I/O opportunities on the FPGA. All 12 GPIO
pins were used, as well as two buttons and two LED's, adding up to 16 bits.

% !TEX root = ../../report.tex
\subsection{Bus interface}
\subsubsection{Overview}

While not strictly a component the databus is still an important part of
our PCB design. The MCU natively supports two bus interfaces, I2C and EBI,
we opted for the latter. We did so because the I2C bus as a 
serial bus has a more limited bandwidth compared to the EBI bus which 
supports 16 bit words. The need for speed stems from the requirement of
streaming at least two audio streams live between the MCU and FPGA.
As an added bonus the EBI bus is compatible with SRAM chip interfaces 
which proved useful when including extra memory in our design.


In addition to the EBI bus we have a special control bus with a width
of 3 signals going between the MCU and FPGA. This bus is available for the
software and FPGA group to use as needed, for instance for interrupts or
other forms of synchronization and status signaling.


Finally, there was an actual connection error in the external RAM footprint.
One of the two chip select inputs, CS1, was suppose to be grounded, but had
mistakenly been connected to VCC33, resulting in the unit being stuck in off
mode. To solve this, the chip select pin was bent slightly upwards and then
connected with a patch cable to a GND pin nearby. However, due to the bus
workaround the SRAM was completely disabled and not used.



\clearpage
\section{Test Results}
Following are the test results of the FR tests, the simulation tests of
\textit{ChaosM}, and finally the power measurement test.

\subsection{FR test results}
Following are the results of the FRs. Due to \textit{ChaosM} not being
successfully implemented, some of the FRTs have not been tested.

\paragraph{Functional requirement 1}
\textit{The audio pipelines must receive and execute different data and instruction streams.}\\
This test was not successfully tested due to the lack of an implemented \textit{ChaosM} on the FPGA chip.

\paragraph{Functional requirement 3}
\textit{The audio processor must consist of at least two audio channels.}\\
One audio pipeline, e.g. an audio channel, was successfully achieved in simulation. Again, this was not confirmed on the system due to \textit{ChaosM}'s lack of successful implementation the system.

\paragraph{Functional requirement 6}
\textit{The cores in the audio pipelines should share data with each other through a ring buffer.}\\
Again, as the above FRT, this was achieved in simulation. It was not tested on the system, and can therefore not be claimed to be achieved.

\paragraph{Functional requirement 7}
\textit{It must be possible to observe power consumption over certain components.}\\
This functional requirement was successfully achieved. With current sensors on the power supply to the MCU and FPGA chips, the results these sensors produced were as expected.

\paragraph{Functional requirement 8}
\textit{The audio system should perform real time filtering on two distinct
channels, with an audio quality of 8-bits and a frequency 44.1kHz.}\\
This FR was not achieved. For the same reasons as FRs 1, 3, 6.

\paragraph{Functional requirement 9}
\textit{The microcontroller and the audio processor should communicate via ARMs
External Bus Interface.}\\
Besides the fact that the PCB did not connect the EBI bus' data lines for the FPGA, this FR was achieved with the documented fix described in Section \ref{fixes}.

\paragraph{Functional requirement 11}
\textit{The MCU must have an SD card interface and a UART serial port.}\\
This FR was successfully achieved. The I/O controller successfully interfaced with the SD card and the UART serial port.

\paragraph{Functional requirement 12}
\textit{The board should have buttons and LEDs.}\\
With the I/O controller, this FR was tested and successfully achieved.

\paragraph{Functional requirement 13}
\textit{The system should be able to process both digital and analog signals.}\\
This FR was partially achieved. While \textit{ChaosM} was as previously mentioned never implemented successfully on the system, the I/O controller successfully implemented demonstration programs that were able to receive and process data from both analog input, the minijacks, and the digital input, the SD card.
In essence, the requirement of being able to utilize both analog and digital sources, was successfully achieved through the use of the ADC and the DAC. This was not achieved on the fully implemented system though.

%The below input is a subsection of the above section

\subsection{FPGA Simulation Results}
While testing \textit{ChaosM} with the aim of accomplishing FR3 
some problems were encountered. These tests show post-route simulation on our 
toplevel vhdl design. This includes the ebi controller, clock controller and the pipeline(s). 
The tests looked like this:

\begin{itemize}
\item Load a program into the processors’ instruction memory. The program will load from its input buffer and store to its output buffer.\\
\item Load input into the first core’s input buffer. The data loaded was “BEEF” for the first pipeline and “1EEF” for the second pipeline (if applicable).\\
\item Read the result from the last core output buffer. The result can be read from the signal called “EBI data”.\\
\end{itemize}


\paragraph{First Test}
The first test was one pipeline with four cores. The timing simulation of this test is
 shown in diagram \ref{fig:one_pipe_four_core}. As the diagram shows, 
 the simulation outputs the correct results shown by signal in orange at the point 
 marked by the yellow, vertical line. 
 
 \begin{figure}[H]
    \centering
    \includegraphics[width=1\textwidth]{figures/fpga/result_1_pipe_4_cores.png}
    \caption{Result from timing simulation of one pipeline with four cores.}
    \label{fig:one_pipe_four_core}
\end{figure}

\paragraph{Second Test}
The second test was two pipelines with four cores in each. The timing simulation 
of this test did not work as expected. Out of the two pipelines simulated only one 
gave correct results, see diagram \ref{fig:two_pipe_four_core}. The first core 
outputted only zeros, as the yellow line shows,  while the second pipeline outputted the correct result “1EEF”. 

\begin{figure}[H]
    \centering
    \includegraphics[width=1\textwidth]{figures/fpga/result_2_pipe_4_cores.png}
    \caption{Result from timing simulation of two pipelines with four cores each.}
    \label{fig:two_pipe_four_core}
\end{figure}


\subsection{FPGA Configuration Results}
Even though one pipeline with four cores simulated correctly in ModelSim, the configured FPGA 
did not work as in the simulation. It behaved nondeterministically with little to no pattern
in the outputted results. A lot of debugging was done and it seems as though the problems
lie in the communication between the microcontroller and the FPGA. 



\clearpage
% !TEX root = ../../report.tex
\section{Power Measurements}

Since this is the report of the energy efficiency group, all components of
system the have been chosen with energy efficiency in mind. In order to measure
the efficiency of our design decisions, a set of power measurements were
performed. This section outlines the results of those measurements.

\begin{table}
    \begin{tabular}{lllll}
	State & FPGA                         & Audio source & Sense3v3 & Sense1v8 \\
	\hline \\
	A     & Idle                         & -            & 140mA    & 75mA \\
	B     & Computational simple filters & ADC          & -        & - \\
	C     & Computational heavy filters  & ADC          & -        & - \\
        D     & Computational simple filters & SD-card      & -        & - \\
        E     & Computational heavy filters  & SD-card      & -        & -
    \end{tabular}
    \label{tab:results/power-measurement}
    \caption{FILL OUT CAPTION}
\end{table}

\todo{Preliminary results, will be fleshed out once we have a running system}


%The below input is a subsection of the previous section
\subsubsection{Power comparision using energy profiling}

The example provided by Silicon Labs named {\bf preamp} is comparable to the main
DMA dataflow, as their net work is similar. Both program uses DMA and moves samples 
from ADC to DAC. The difference is that the non-example code includes deinterleaving
and interleaving of the audio channels. Refere to section \ref{} for details. Figures \ref{fig:fpga} and \ref{fig:preamp}
shows the energy consumption by the Microprocessor when running each program.

\begin{figure}[H]
    \centering
    \includegraphics[width=250px]{figures/sw/fpga.png}
    \caption{The enegy usage for DMA}
    \label{fig:fpga}
\end{figure}


\begin{figure}[H]
    \centering
    \includegraphics[width=250px]{figures/sw/preamp.png}
    \caption{Energy usage for the example code}
    \label{fig:preamp}
\end{figure}


As the figures suggests the energy consumption for each program are practically equal.

