\FloatBarrier
\chapter{FPGA}\label{chapter:fpga}

As this project requires us to implement a MIMD processor in VHDL, we realized
our design through implementing it in VHDL, and running the final processor on a
Spartan-6\cite{fpga-chip} FieldProgrammableGateArray.

Our MIMD processor design is was realized through pipelined cores connected
sequentially in a pipeline of their own. Since the processor consists of
homogenous cores, each core can receive and run the same instructions. To
realize this as a ``Multiple Instructions, Multiple Data'' (``MIMD''), the first
core in the sequential core-pipeline is the only one which can work with the
data given to said pipeline by the EBI bus. The next core then works on the
output of the first core, and in this fashion each core in the pipeline works on
its own data independently of the data any other core is currently working
on\footnote{This is how we realize the ``Multiple Data'' part of the MIMD
definition.}.

And each of the processor cores has its own instruction memory, so that they can
run each their own instructions, and that's how we have realized the ``Multiple
Instruction'' part of the MIMD definition.
\newpage
\FloatBarrier
\section{Design Planning}\label{section:fpga-design}
\subsection{Design start}

As soon as it became decided that our project was going to aim for producing
some kind of ``sound-effect manipulating processor'', the group started looking
into how sound-effects were actually accomplished.

It quickly became apparent that for the most common, as well as the more
complicated effects, Fourier-transforms were crucial to the sound-manipulation
process. At least when sound manipulation was intended to run quickly enough,
as well as successfully on a computer processor.

This represented an (until now) unforeseen problem: Fourier-transforms are
algorithmically heavy in both time complexity and storage-space complexity,
which has the consequence that we would have to devote more of the FPGA
resources into circuits which could handle something as intensive as a
Fourier-transform. Especially if we wanted to perform these sound-effects in
real-time.

In effect, this revelation forced us to from a very early point to be
concerned about resource management on the FPGA.

\subsection{Fourier-transforms}\label{subsection:fpga-design-ft}

Looking into which types of Fourier-transform algorithms we could use on an
embedded device, in a manner as efficiently as possible, and yet real-time,
we ended up looking at Sliding-Discrete-Fourier-Transforms\cite{SD-FT} (or more
colloquially known as SD-FTs\footnote{This is discussed in more detail in
section \ref{subsubsection:fpga-alu-ft}.}).
\missingfigure{Maybe insert a figure here showing how the SD-FT works on each
sample at a time?}

This Fourier-transform would enable the FPGA to receive a live stream of
data samples representing sound from an input device through a mini-jack on the
PCB, or just from the DMAs of the MCU if the input was a file on a SD-card. And
then transform this live stream of samples into the frequency domain.

Since we now were able to work on each discrete sample while it was converted
into the frequency domain, we were now able to perform frequency based
sound-manipulations, such as basic high-pass and low-pass filters.

The sound-effect manipulations could now (in theory) be put to work.

\FloatBarrier
\subsection{Sequential processing cores?}\todo{Maybe find better subsection
title?}

Not all of the known sound-effects require (or are even possible) with a
Fourier-transform. Therefore it also became evident that for effects like say
the ``echo-effect'', we would need to do some post-processing after having
performed the inverse-Fourier-transform of the SD-FT\footnote{Sliding-Discrete
Fourier-transform, elaborated in the previous section
\ref{subsection:fpga-design-ft}.}.

Thus, so far this project required one processor-core to perform the SD-FT, and
another to perform the actual sound-effect done with the samples in the
frequency domain. A third core for the inverse SD-FT, and finally a fourth core
for any possible sound-manipulation that does not need a Fourier-transform (that
does not need to manipulate the data samples when converted into the frequency
domain).

With that reasoning in mind, we would need at least four processing cores, and
this for only one sound-channel. So for every additional sound-channel needed (
stereo needs two), we would have to add another four processor cores onto the
FPGA-chip.

It was decided early on to not settle for just one channel producing mono sound,
but to instead settle for at least two, so that we could listen to the output
on a stereo system. This let us then define the design for the processor
further, since it would need to have at the very least eight processor cores
implemented in some form of pipeline or other.

\subsection{Processing cores, 1-on-1}\todo{Maybe find better subsection title?}

Knowing we would need a minimum of eight processor cores on our FPGA processor,
and maybe more later down the line, we decided to start making things easier
for ourselves.

The decision was made to have all the cores homogenous, so that it would be
possible to send the same instructions to any core, being safe in the knowledge
that any core could perform said instructions.

This simplified the work in VHDL considerably, since we could then focus on
getting just \emph{one} processor core functioning, instead of say one
specifically for the SD-FT, another specifically tailored for the actual
sound-effects performed in the frequency domain, a third for the inverse-SD-FT,
and then the same as well for the fourth and final core performing the
``post-processing''. This would have (in all likelyhood) more than quadrupled
the amount of work required in VHDL for this project.

Due to all of the members in the FPGA-group taking the course TDT4255 \emph{
Computer Design} it may come as no surprise to any reader familiar with the
courses at IDI, NTNU, that this course was a big source of inspiration for the
development of our processor-core. Both the book\cite{tdt4255-book} and the
\emph{Computer Design} course were used as resources while developing the design
of this MIPS pipelined processor.

\subsection{Toplevel control register}\label{subsection:fpga-design-toplevel}


\FloatBarrier
\subsection{Audio Pipelines}\label{subsec:audio_pipelines}

\missingfigure{Use the same diagram as below, but this picture needs
the same amount of cores, and the same programs, as the demo will run!}

\begin{figure}[H]
    \centering
    \includegraphics[height=150px]{figures/fpga/system_components_general_pipeline.png}
    \caption{Audio Pipeline Architecture}
    \label{fig:pipeline_architecture}
\end{figure}

\textit{ChaosM} consists of several audio processing pipelines, as illustrated in
figure \ref{fig:pipeline_architecture}. These contain several processing cores
separated by data buffers. Each audio processing pipeline processes one channel
of sound data.


\FloatBarrier
\section{Processor Core}\label{section:fpga-processor-core}
\todo[inline]{How is the processor designed and \emph{why!}}

\FloatBarrier
\subsection{Considerations}
\todo[inline]{Describe the pipelines of the cpu, and why it is implemented/
designed the way it is.}

The processor core was implemented with focus on three aspects:

\begin{enumerate}
	\item Real-time sound manipulation
	\item A focus on energy-efficiency.
	\begin{enumerate}
		\item Since we are implementing this on an FPGA, the focus will be on
simulating (through having signals in VHDL simulating energy-efficiency switch
choices) opportunities and the potential for energy-efficiency. This because
it will be most impossible to program and test an FPGA chip not meant
specifically for energy-efficiency purposes to be tested with respect to
energy-efficiency.
	\end{enumerate}
	\item An architectural design permitting the manipulation of the most common
sound-effects on the above real-time datastream, and yet still having such a
generalized design so as to permit the running of most standard MIPS
instructions.
\end{enumerate}

With these three points in mind, the decision was made to base the design on a
processor very similar to the generalized pipelined MIPS processor that the
subject TDT4255\cite{tdt4255} has its students implementing in the course
exercises.

\FloatBarrier
\FloatBarrier
\subsection{ALU}\label{subsec:fpga-alu}

Which operations were implemented in the ALU was decided based on behalf of what
was neccessary for the Fourier-transform, sound processing, and on whether the
instruction set had room for the encoding of the operations.

Division was not implemented in hardware in order for all instructions to be
single-cycle. However, division can be done using floating point multiplication
or by a regular long division algorithm in software.

The ALU implementation receives two 16-bit registers as inputs, and outputs a
32-bit register. However, only the lower 16-bits are currently used. Previously,
the intention was for multiplication operations to output all 32-bits.

\subsubsection{Internal memory}

Since the instruction words can only address two registers at the time, the ALU
unit can store the values previously loaded from constant memory internally.
``Multiply and accumulate'' is the longest critical path, and which is used
during the SD-FT \ref{appendix:sd-ft}. This instruction needs the constant value
previously loaded into the ALU with the ``Load from constant memory''
instruction. Hence why the ALU can store a constant memory value internally.


\FloatBarrier
\subsection{Floating-point implementation design choices}

\FloatBarrier
\subsection{Memory access}
\todo[inline]{Why constant values go to the alu, and how it changes pipeline
layout. How the core accesses memory.}
%Should contain: Ringbuffer, Switch buffer, input/output buffers, constant memory, inst memory, read and write lines to the fpga from the ARM. Originally the ARM only needed access to the first and the last data buffer but for debugging purposes the ARM was given access to all buffers.
\section{Internal memory\todo{Better title}}\label{section:fpga-internal-memory}
\todo[inline]{Short description? Discuss non-uniform memory access and important things.}

\subsection{Intruction memory}
\todo[inline]{Describe the functionality and reasons behind decisions regarding the instruction memory.}

\subsection{Processor input/output memory}
\todo[inline]{Describe the need for memory to behave as both ringbuffer/queues, and as switching buffers. Discuss why this is a good idea because of our intended functionality, and how we utilize the FPGAs resources to implement it in the best possible way.}

\subsection{Pipeline constant memory}
\todo[inline]{Why did we separate this from input/ouput memory? Why is it shared between cores, and how? The reason we send it's values directly to the alu instead of to the register file?}
\FloatBarrier
\section{Communication}\label{section:fpga-buses}

\subsection{The External Bus Interface}
The microcontroller communicates with the FPGA design using the External Bus
Interface (EBI) of the Giant Gecko microcontroller. The EBI is a parallel bus
with a separate data and address bus in addition to the chip select and read and
write enable signals, all active low\cite{efm_ebi}.

The communication between the MCU and the FPGA uses 23 address lines and 16
data lines. All transfers are initiated when the chip select signal goes low.
For write transfers, the data and address lines are then set up and the
write enable signal is asserted, see figure \ref{fig:ebi_write}. For reads,
the address is set up and the data line is put in high impedance mode before
the read enable signal is asserted, see figure \ref{fig:ebi_read}.

\begin{figure}[h]
	\centering
	\includegraphics[width=0.8\linewidth]{figures/fpga/ebi_write.png}
	\caption{EBI write transfer\cite[p.6]{efm_ebi}}
	\label{fig:ebi_write}
\end{figure}



\begin{figure}[h]
	\centering
	\includegraphics[width=0.8\linewidth]{figures/fpga/ebi_read.png}
	\caption{EBI read transfer\cite[p.7]{efm_ebi}}
	\label{fig:ebi_read}
\end{figure}



\FloatBarrier
\subsection{The Internal Bus}

The internal bus is used to transfer data to and from modules in the FPGA.
All transfers are initiated by the microcontroller on the EBI bus, and the
EBI controller facilitates communication between the EBI and the
internal bus.

\subsubsection{The EBI controller}
The EBI controller module is used to handle EBI transfers initiated by the
microcontroller. It consists of a simple state machine, illustrated in
figure \ref{fig:ebi_ctrl_fsm}. When an EBI transfer is executed, the
internal bus, described in is used to store or retrieve data from a module
in the FPGA.

The EBI controller keeps the EBI data lines in high impedance mode whenever
a transfer is not active, allowing the MCU to have control of the EBI bus.

\begin{figure}[h]
	\centering
	% TODO: Make the arrows in separate directions be separate arrows
	\begin{tikzpicture}[shorten >= 1pt,node distance=2cm,on grid,auto]
		\node[state,initial] (idle) {idle};
		\node[state] (read) [above right=of idle] {read};
		\node[state] (write) [below right=of idle] {write};
		\path[->]
			(idle) edge node {\small RE = 0} (read)
			       edge node [swap] {\small EW = 0} (write)
			(read) edge node {\small RE = 1} (idle)
			(write) edge node [swap] {\small WE = 1} (idle);
	\end{tikzpicture}
	\caption{EBI controller state machine}
	\label{fig:ebi_ctrl_fsm}
\end{figure}




\FloatBarrier
\subsubsection{Addressing}

Modules in the FPGA is addressed using a simple addressing scheme, where the
address is divided into several parts, as illustrated in figure \ref{fig:ebi_addresses}.

\begin{figure}[h]
	\centering
	\begin{bytefield}[bitwidth=0.04\linewidth]{23}
		\bitheader{0-22}\\
		\bitbox{1}{T} &
		\bitbox{2}{\tiny Pipeline} &
		\bitbox{4}{Device} &
		\bitbox{2}{\tiny Subdev} &
		\bitbox{14}{Address}
	\end{bytefield}
	\caption{FPGA address format}
	\label{fig:ebi_addresses}
\end{figure}




In an EBI address, the T bit is used to select the toplevel control register.
If the T bit is set, the rest of the address is ignored, and only the toplevel
control register is accessible.

If the T bit is not set, the pipeline field is used to select which pipeline
to address. In the pipeline, the device field is used to select
which pipeline module to address. Two device numbers have special meaning;
device 0 is the pipeline control register, while device 1 is the constant
memory. All device numbers starting at 2 accesses the cores in the pipeline.

The subdevice field is used to select between modules in each core. These
are listed in table \ref{tab:core_subdevices}.

\begin{table}[h]
	\centering
	\begin{tabular}{|l l|}
		\hline
		\textbf{Number} & \textbf{Description} \\
		\hline
		0 & Control register \\
		1 & Instruction memory \\
		2 & Input buffer \\
		3 & Ouput buffer \\
		\hline
	\end{tabular}

	\caption{Processor core subdevices}
	\label{tab:core_subdevices}
\end{table}


\FloatBarrier
\subsubsection{Read Transfers}

A read transfer is initiated when the read enable line from the microcontroller
goes low. The EBI controller sets up the internal address signals and asserts
the internal read enable signal. This causes the requested data to be available
in the next clock cycle. The EBI controller switches to read state, where it
remains until the chip select signal is deasserted.

\subsubsection{Write Transfers}

Write transfers are initiated the same way as read transfers. As the 
write-enable signal goes low, the destination address is latched into the 
internal address bus and the internal data lines are set to the value of the 
EBI data lines. The EBI controller enters write state and the internal
write enable signal is asserted. The idle state is reentered when the chip
select is deasserted.


% !TEX root = ../../report.tex
\subsection{Instruction Set Architecture}\label{section:fpga-isa}

The processor was designed in a top-down fashion, starting with
the instruction set. In order to support as many different filters
and effects as possible, the processor supports all normal arithmetic
operations. Due to the requirements of doing fourier transforms on the
processor, support for some floating point instructions was also included.

To make the decoding of instructions as easy as possible, instructions
were divided into three different instruction groups.

\subsubsection{Register-based Instructions}

The register-based instructions are instructions were both operands are
primarily registers. The group also includes a few instructions where
the second operand is an immediate value. The format of the
instructions are illustrated in figure \ref{fig:regbased_instrs_format}. The
implemented functions can be found in table \ref{tab:regbased_instrs}.

\todo{Fiks instruksjonstabellen}

\begin{figure}[h]
	\centering
	\begin{bytefield}[endianness=big,bitwidth=0.05\linewidth]{16}
		\bitheader{0-15}	\\
		\bitbox{2}{Group}	&
		\bitbox{2}{Funct}	&
		\bitbox{2}{Opt}		&
		\bitbox{5}{Reg A}	&
		\bitbox{5}{Reg B/Imm}
	\end{bytefield}

	\caption{Register-based instruction format}
	\label{fig:regbased_instrs_format}
\end{figure}

\begin{table}[H]
	\centering
	\begin{tabular}{|l l l l l|}
		\hline
		\textbf{Funct} & \textbf{Opt}  & \textbf{Mnemonic} & \textbf{Instruction} & \textbf{Operation} \\
	\hline
	\multicolumn{5}{|c|}{Group \texttt{0b00}} \\
	\hline
	\multirow{3}{*}{\texttt{0b00}}
		& \texttt{0b00} & \texttt{add \$ra, \$rb}  & Add registers & $\$ra \leftarrow \$ra + \$rb$ \\
		& \texttt{0b01} & \texttt{addi \$ra, imm} & Add immediate & $\$ra \leftarrow \$ra + imm$ \\
		& \texttt{0b10} & \texttt{fadd \$ra, \$rb} & Add registers (FP) & $\$ra \leftarrow \$ra + \$rb$ \\
	\multirow{4}{*}{\texttt{0b01}}
		& \texttt{0b00} & \texttt{sub \$ra, \$rb}  & Subtract registers & $\$ra \leftarrow \$ra - \$rb$ \\
		& \texttt{0b01} & \texttt{fsub \$ra, \$rb} & Subtract registers (FP) & $\$ra \leftarrow \$ra - \$rb$ \\
		& \texttt{0b10} & \texttt{cmp \$ra, \$rb}  & Compare & $cnd \leftarrow cnd(\$ra - \$rb)$ \\
		& \texttt{0b11} & - & compare fp / subtract imm? & - \\
	\multirow{4}{*}{\texttt{0b10}}
		& \texttt{0b00} & \texttt{mul \$ra, \$rb}  & Multiply registers & $\$ra \leftarrow \$ra * \$rb$ \\
		& \texttt{0b01} & \texttt{fmul \$ra, \$rb} & Multiply registers (FP) & $\$ra \leftarrow \$ra * \$rb$ \\
		& \texttt{0b10} & \texttt{fmla \$ra, \$rb} & Multiply-and-accumulate (FP) & $\$ra \leftarrow \$ra + \$rb * \$rc$ \\
		& \texttt{0b11} & \texttt{fmls \$ra, \$rb} & Multiply-and-subtract (FP) & $\$ra \leftarrow \$ra - \$rb * \$rc$ \\
	\multirow{4}{*}{\texttt{0b11}}
		& \texttt{0b0x} & \texttt{halt} & Stop processor &  \\
		& \texttt{0b10} & \texttt{fmla \$ra, \$rb} & Multiply-and-accumulate (FP) & $\$ra \leftarrow \$ra + \$rb * \$rd$ \\
		& \texttt{0b11} & \texttt{fmls \$ra, \$rb} & Multiply-and-subtract (FP) & $\$ra \leftarrow \$ra - \$rb * \$rd$ \\
	\hline
	\multicolumn{5}{|c|}{Group \texttt{0b01}} \\
	\hline
	\multirow{2}{*}{\texttt{0b00}}
		& \texttt{0b00} & \texttt{and \$ra, \$rb} & And & $\$ra \leftarrow \$ra \wedge \$rb$ \\
		& \texttt{0b01} & \texttt{nand \$ra, \$rb} & Nand & $\$ra \leftarrow \neg(\$ra \wedge \$rb)$ \\
	\multirow{3}{*}{\texttt{0b01}}
		& \texttt{0b00} & \texttt{or \$ra, \$rb} & Or & $\$ra \leftarrow \$ra \vee \$rb$ \\
		& \texttt{0b01} & \texttt{nor \$ra, \$rb} & Nor & $\$ra \leftarrow \neg(\$ra \vee \$rb)$\\
		& \texttt{0b10} & \texttt{xor \$ra, \$rb} & Xor & $\$ra \leftarrow \$ra \oplus \$rb$\\
	\multirow{4}{*}{\texttt{0b10}}
		& \texttt{0b00} & \texttt{mov \$ra, \$rb} & Move & $\$ra \leftarrow \$rb$\\
		& \texttt{0b01} & \texttt{mvn \$ra, \$rb} & Move negative & $\$ra \leftarrow \neg\$rb$ \\
		& \texttt{0b10} & \texttt{i2f \$ra, \$rb} & Typecast (Int to FP) & $\$ra \leftarrow fp(\$rb)$ \\
		& \texttt{0b11} & \texttt{f2i \$ra, \$rb} & Typecast (Fp to int) & $\$ra \leftarrow int(\$rb)$ \\
	\multirow{4}{*}{\texttt{0b11}}
		& \texttt{0b00} & \texttt{lda \$ra, [\$rb]} & Load from input & $\$ra \leftarrow [\$rb]$ \\
		& \texttt{0b01} & \texttt{ldb \$ra, [\$rb]} & Load from output & $\$ra \leftarrow [\$rb]$ \\
		& \texttt{0b10} & \texttt{ldc \$ra, [\$rb]} & Load from constant buffer & $\$rd\ and\ \$rc \leftarrow [\$rb]$ \\
		& \texttt{0b11} & \texttt{stb \$ra, [\$rb]} & Store to output & $[\$rb] \leftarrow \$ra$ \\
	\hline
	\end{tabular}
	\caption{Register-based instruction list}
	\label{tab:regbased_instrs}
\end{table}



\subsubsection{Load Immediate Instruction}
The load immediate instruction is used to load an immediate constant into a
register. The format of the instruction can be found in figure
\ref{fig:ldi_format}. The value is loaded into register \texttt{\$r1}.

\begin{figure}[h]
	\centering
	\begin{bytefield}[endianness=big,bitwidth=0.05\linewidth]{16}
		\bitheader{0-15} \\
		\bitbox{2}{Group} &
		\bitbox{14}{Imm}
	\end{bytefield}

	\caption{Load immediate format}
	\label{fig:ldi_format}
\end{figure}


\subsubsection{Branch Instruction}
The branch instruction checks condition flags and jumps accordingly. By
checking for various combinations of the condition flags, many different
conditions can be checked for. The format of the instruction is illustrated
in figure \ref{fig:new_branch_format}. List for the encoding of the flags field
is coming later.

\begin{figure}[h]
	\centering
	\begin{bytefield}[endianness=big,bitwidth=0.05\linewidth]{16}
		\bitheader{0-15} \\
		\bitbox{2}{Group} &
		\bitbox{4}{Flags} &
		\bitbox{11}{Target}
	\end{bytefield}

	\caption{Branch instruction format}
	\label{fig:new_branch_format}
\end{figure}



\paragraph{Special Registers}

Due to the limited space in instruction words, only two registers at most can be
specified in an instruction. Some instructions, such as the load immediate and
load constant instructions do not specify any registers, only an immediate
offset constant.

The list of defined special registers can be found in table \ref{tab:specregs}.
The register number for these registers have not yet been determined.

\begin{centering}[H]
	\centering
	\begin{tabular}{|l p{10.5cm}|}
		\hline
		\textbf{Register name} & \textbf{Register purpose} \\
		\hline
		\texttt{r0} & Zero register, hard coded to always contain 0 \\
		\texttt{r1} & Immediate register, contains the result of an \textsc{ldi} instruction\\
		\texttt{rc} & Constant register, provides operand to \textsc{fmla} and \textsc{fmls} instructions. Lies within the ALU. \\
		\texttt{rc} & Constant register, provides operand to \textsc{fmla} and \textsc{fmls} instructions. Lies within the ALU. \\
		\hline
	\end{tabular}
	caption{List of special registers}
	\label{tab:specregs}
\end{centering}




\section{Testing the FPGA}
\textsl{•}

\subsection{FPGA prototype}
\todo[inline]{How we tested that the FPGA was soldered onto the PCB. Can move this section.}

\subsection{Testing the processor design}
\todo[inline]{Testing methodology}

\subsection{Simulation}
\todo[inline]{Simluation description}

\subsubsection{Tests\todo{insert more of these, along with descriptions and reasons.}}

\subsection{Integrated tests}
\todo[inline]{Description of testing on board}

\subsubsection{Tests\todo{insert more of these, along with descriptions and reasons.}}


