% !TEX root = ../../report.tex

\section{Schematics}

The development tool used to design the PCB was Altium Designer 2013. Each sheet
contains components relevant to what is routed in it, which means that large
components that consist of several blocks, such as the FPGA and MCU, are spread
out across different schematics. This has been done because we found it more
intuitive to have as many involved components as possible on the same sheet
instead of connecting loads of cables across sheets. The schematics themselves
can be viewed in Appendix~\ref{apx:schematics} \todo{review, rewrite}

From early on we were strongly advised to make things as easy as possible for
ourselves, which meant that whenever we found premade drawings or schematics of
something we wanted to implement that had been tested and were known to work, we
should copy it. So we did. Mostly we used schematics from datasheets, but we
also looked at what the energy efficiency group of 2012 had done before us on
their power supply, in addition to double checking this with the datasheets for
the relevant components.

We also focused on adding backup solutions and debugging options in case
something would fail due to PCB errors.\todo{rewrite, add more about being able
to fall back on other protocols should one fail, and the mapping of unused
pins..}

% !TEX root = ../../../../report.tex

\subsubsection{Programming Interfaces}

\paragraph{FPGA Programming}
The FPGA can be programmed in numerous configurations, all of which have
distinct features and capabilities described in the \todo{insert reference?}
Spartan-6 FPGA Configuration User Guide. To avoid ruining the programmability, a
decision was made to go for the simplest and less error-prone configuration; a
JTAG interface directly to the chip.

However, JTAG programming does not provide persistence -- the FPGA would have to
be reprogrammed every time it powered up. This was not an ideal solution, so a
FPGA configuration flash memory was added, from which the FPGA could read its
initial configuration at startup. Xilinx provides schematics to daisy-chain the
regular JTAG with a JTAG interface to the FPGA flash, making it easy to
implement. Using this setup both the FPGA and the flash will be programmed
simultaneously, and unlike the FPGA the flash will retain the program when
powered down. When powered up again the FPGA will look to the flash for its
initial program.

\paragraph{MCU Programming}
Placing programming headers for the MCU on the PCB is a much simpler procedure.
The 20-pin ARM debug pinout is well documented in application notes and is
simple to set up.

In addition to programming the MCU, the debug header permits tracing of running
programs, as well as allowing the use of the energy profiler provided by Silicon Labs.
