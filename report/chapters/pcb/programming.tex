% !TEX root = ../../report.tex
\section{Programming Interfaces}

\subsection{FPGA programming}
The FPGA can be programmed in numerous configurations, all of which has distinct
features and capabilities described in the Spartan-6 FPGA Configuration User
Guide. To prevent us from ruining the programmability, we decided to go for the
simplest and less error-prone configuration; a JTAG interface directly to the
chip.

However, JTAG programming does not provide persistence -- the FPGA would
have to be reprogrammed every time it powered up. This was not a very ideal
solution, so we decided to add a FPGA configuration flash memory, from which
the FPGA could read its initial configuration at startup. Xilinx provides
schematics to daisy-chain \todo{Correct Runar?} the regular JTAG with a
JTAG interface to the FPGA flash, making it easy to implement.

\subsection{MCU programming}
Putting out programming headers for the Giant Gecko is a much simpler procedure.
The 20-pin ARM debug pinout is well documented in application notes and is easy
to set up.
