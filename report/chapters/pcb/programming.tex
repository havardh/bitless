% !TEX root = ../../report.tex
\section{Programming Interfaces}

\subsection{FPGA programming}
The FPGA can be programmed in numerous configurations, all of which has distinct
features and capabilities described in the Spartan-6 FPGA Configuration User
Guide. To prevent us from ruining the programmability, we decided to go for the
simplest and less error-prone configuration; a JTAG interface directly to the
chip.

However, JTAG programming only would not give us persistence -- the FPGA would
have to be reprogrammed each time the power was plugged in. This was not
satisfying, so we decided to add a FPGA configuration flash memory the FPGA
could read it's initial configuration from at startup. Xilinx provided
schematics to daisy-chain \todo[inline]{Correct Runar?} the regular JTAG with a
JTAG interface to the FPGA flash, making it easy to implement.


\subsection{MCU programming}
Putting out programming headers for the Giant Gecko is a much simpler exercise.
The 20-pin ARM debug pinout is well documented through application notes and is
easy to set up.
