% !TEX root = ../../report.tex
\section{Power Supply}

Being on the energy efficiency group, maximum power utilization has been one of our highest priorities. Therefore, we had to make sure that as little power as possible goes to waste during the transformation from whatever input we get to the 3.3V and 1.2V the components on our chip will be using. Before deciding on a power supply design we looked at what the previous groups had done, especially the energy efficiency group of 2012, as they faced much of the same challenges as we have. Anything before 2012 had been using the same power supply implemented by one of the earliest groups taking this class, which was a very simple design, but not very efficient in terms of power saving. It was basically built to transform any excess power into heat, thus utilizing only a fraction of the input power for running the chip.

Fortunately for us, the energy efficiency group of 2012 did a solid job on their power supply, utilizing hardware parts that were able to output 3.3V from any input voltage staring at 4.75V up to 18V with an efficiency of 83-90\%, see Appendix~\todo[inline]{add appendix sr10s3v3}. So we copied theirs and made some small adjustments.

The main issue with the original design was that the current sensors were not sensitive enough for low voltages, which resulted in inaccurate measurements. To remedy this we found a more suitable current sensor for our needs. Finding the right one took some time, because current sensors come in a variety of shapes and sizes, in terms of supply voltage and sensitivity. In our case, we needed one that could run on 3.3V and sense currents of the same range. Most current sensors need at least 5V supply, or were not sensitive enough.\todo[inline]{verify or re-write} 



  



	
