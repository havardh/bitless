% !TEX root = ../../report.tex
\subsection{Programming Interfaces}

\subsubsection{FPGA programming}
The FPGA can be programmed in numerous configurations, all of which has distinct
features and capabilities described in the Spartan-6 FPGA Configuration User
Guide. To prevent us from ruining the programmability, we decided to go for the
simplest and less error-prone configuration; a JTAG interface directly to the
chip.

However, JTAG programming does not provide persistence -- the FPGA would
have to be reprogrammed every time it powered up. This was not a very ideal
solution, so we decided to add a FPGA configuration flash memory, from which
the FPGA could read its initial configuration at startup. Xilinx provides
schematics to daisy-chain the regular JTAG with a
JTAG interface to the FPGA flash, making it easy to implement. Using this
setup both the FPGA and the flash will be programmed simultaneously, and 
unlike the FPGA the flash will retain the program when powered down. When
powered up again the FPGA will look to the flash for its initial program.

\subsubsection{MCU programming}
Putting out programming headers for the Giant Gecko is a much simpler procedure.
The 20-pin ARM debug pinout is well documented in application notes and is easy
to set up.

In addition to programming the MCU, the debug header allows us to trace running
programs \todo{Håvard?} and make use of the energy profiler provided by 
Energymicro.