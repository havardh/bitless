% !TEX root = ../../report.tex

\section{Fixes}
\todo{Quick notes taken in the lab, needs a rewrite probably}

As the group had never before designed a PCB before, and since there are so
many things that can go wrong, obviously something was bound to not 
work as intended intially.

After receiving the PCB we quickly realised that the footprint for the 
linear regulator was miss-matched with the pinout of the component. 
After consulting our design project we realised that the footprint had
been fixed early on, but we had forgot to update the component within 
the schematic. The same turned out to have happend with the low frequency
crystal connected to the MCU.

Both the above problems where easly solved. For the crystal pin 1 and 4
was suppose to be used, but we had 1 and 2 in our design. As 2 and 3 are 
unconnected we where able to solve the problem by soldering a patch cable
between pin 4 and 2. \todo{Photo} 

The linear regulator was slightly more advanced as we had switched pin
1 and 5, and pin 2 and 4 in our design. The solution was soldering six 
short patch cables on the pad itself, and then mount the linear regulator
to those cables after ordering them correct. The resulted in the linear 
regulator being mounted about 1 cm above the card. \todo{photo, both physcal and a schematic maybe?}

In addition to the two footprints we had forgot to update we also had
an actual connection error in the SRAM footprint. One of the two chip
select inputs, CS1, was suppose to be grounded, but we had mistakenly connected
it to VCC33, resulting in the unit being stuck in off mode. To solve
this we slighly bent the chip select pin upwards, and connected it using
a patch cable to an other pin near by that was grounded. \todo{photo}