% !TEX root = ../../report.tex

\section{PCB Test Plan}

In order to ease the process of assembly of the system once the PCB arrived, a
rough step by step test plan was prepared. The purpose of the plan was to catch
any production or design errors as early as possible and to reduce the number of
wasted work hours due to any critical errors that could have been detected early
on.

\begin{enumerate}
    \item Manually test important solder points using a multimeter. This 
    includes connections between the power source and regulators. After 
    checking these points it is safe to assume that the connections within
    the power supply is as expected, and that short circuits due to design
    or production errors can be excluded.
    \item After soldering the power supply the provided voltage levels should
    be measured using the four measuring sockets available for each of the
    provided voltage levels and ground. This is an important step as sensitive
    components might be damaged if the provided levels are wrong. 
    \item After soldering the major components as well as the JTAG and debug
    interfaces, specially prepared test program should be flashed. These will
    validate connections to the outside world, as well as connections between 
    important components. These programs will be provided by the FPGA and 
    Software groups ahead of time.
    \item After soldering the entire board, final integration tests will be run where all
    provided user interfaces are tested. This includes testing of UART over 
    USB, reading and writing from the SD card, as well as checking the audio input and output interfaces. This phase of 
    the testing will be carried out by the software group, again using specially
    prepared test programs.
    \item Finally, after the FPGA and Software group have completed their work,
    or at least completed the design necessary to run any testing, the PCB group
    will begin measuring of the overall power usage of the board. These measurements
    will be used to evaluate the possibility of powering the board over USB, as 
    mentioned in section~\ref{psu:usb} - Power by USB.
\end{enumerate}
