\section{System Testing}
Each group have implemented and tested 

\begin{description}
    \item[Functional requirement 1] \hfill \\
        \textit{The audio cores must consume different data and instruction streams. } \\
        \textbf{Test:} Initialize the audio processor with at least two pipelines
        containing one core. Program the two cores with two different programs and feed
        each pipeline with different data.
        \textbf{Pass:} The cores in each pipeline is able to read data from the input buffer,
        perform operations on the data and write the result to the pipelines output buffer.\\
    \item[Functional requirement 2] \hfill \\
        \textit{The processor cores should be arranged in different pipelines,
        each pipeline representing an audio channel.} \\
        \textbf{Test:} \\
        \textbf{Pass:} \\
    \item[Functional requirement 3] \hfill \\
        \textit{The audio processor must consist of at least two audio channels. } \\
        \textbf{Test:} Initialize the audio processor with two pipelines, where each of the 
        pipelines does calculations on an audio channel. Use a stereo wav-file and send
        the left channel through the first pipeline and the right channel through the
        second one. \\
        \textbf{Pass:} The data received from the the output buffer is equal to the
        data sent to the input buffer. \\
    \item[Functional requirement 4] \hfill \\
        \textit{The processor architecture should utilize 16 bit processing cores. } \\
        \textbf{Test:} \\
        \textbf{Pass:} \\
    \item[Functional requirement 5] \hfill \\
        \textit{The built-in DSP hardware on the FPGA must be utilized efficiently. } \\
        \textbf{Test:} \\
        \textbf{Pass:} \\
    \item[Functional requirement 6] \hfill \\
        \textit{The cores in the audio pipelines should share data with each other} \\
        \textbf{Test:} \\
        \textbf{Pass:} \\
    \item[Functional requirement 7] \hfill \\
        \textit{It must be possible to observe power consumption over certain components.} \\
        \textbf{Test:} \\
        \textbf{Pass:} \\
    \item[Functional requirement 8] \hfill \\
        \textit{The audio system should perform real time filtering on} \\
        \textbf{Test:} \\
        \textbf{Pass:} \\
    \item[Functional requirement 9] \hfill \\
        \textit{The microcontroller and the audio processor should communicate via ARMs} \\
        \textbf{Test:} \\
        \textbf{Pass:} \\
    \item[Functional requirement 10] \hfill \\
        \textit{The audio processor should be able to perform Hardware-accelerated FFTs.} \\
        \textbf{Test:} \\
        \textbf{Pass:} \\
    \item[Functional requirement 11] \hfill \\
        \textit{The MCU must have an SD card interface and a UART serial port.} \\
        \textbf{Test:} Copy a file to the SD card with the MCU, read the data on the SD card
        from an external device. Connect to the board with UART and write to an external device.\\
        \textbf{Pass:} The file is copied correctly and the external device receives the 
        transmitted data. \\
    \item[Functional requirement 12] \hfill \\
        \textit{The board should have buttons and LEDs.} \\
        \textbf{Test:} \\
        \textbf{Pass:} \\
    \item[Functional requirement 13] \hfill \\
        \textit{The audio system should be able to process both digital and analog signals.} \\
        \textbf{Test:} Run the audio processor with input data from a Micro SD card, and
        with data from an ADC. \\
        \textbf{Pass:} The PCB is equipped with both a Micro SD port and and ADC, and the
        MCU is able to fully control both of the peripherals.\\
\end{description}


\subsubsection*{Functional Requirement 1}
\textit{The processors should consume different data and instruction stream}
\subsubsection*{Test:}
Create a program that requires cores to run multiple instructions on multiple data. 
Run this program in ModelSim to verify that the cores simulate correctly. 
Finally, run the same program on the FPGA embedded in the system and verify that it produces the same results as the simulation.
\subsubsection*{Pass Criteria:}
The cores run different instructions on different data and produce the correct results according to the program. 


\subsubsection*{Functional Requirement 2}
\textit{The processor cores should be arranged in a pipeline, each pipeline representing one audio channel. Filter at least two audio channels.}
\subsubsection*{Test:}
Create a program that requires two pipelines with at least one core in each pipeline. 
Simulate the program in ModelSim and verify that the output reflects the desired results.
Run the program on the FPGA embedded in the system and verify that the result is the same as in the simulation.
\subsubsection*{Pass Criteria:}
At least two pipelines run one audio channel each and produce the correct results according to the program.

\subsubsection*{Functional Requirement 3}
\textit{Use the built-in DSP hardware efficiently.}
\subsubsection*{Test:}
Synthesize one processor core and review the synthesis report. 
\subsubsection*{Pass Criteria:}
There are few/none latches and an appropriate amount of DSP slices are used.


\subsubsection*{Functional Requirement 4}
\textit{Conveniently observe power consumption over certain components.}
\subsubsection*{Test:}
\subsubsection*{Pass Criteria:}


\subsubsection*{Functional Requirement 5}
\textit{Real time filtering on two distinct channels, audio quality (8-bit 44.1kHz).}
\subsubsection*{Test:}
\subsubsection*{Pass Criteria:}

\subsubsection*{Functional Requirement 6}
\textit{The processor architecture shall utilize 16 bit processing cores.}
\subsubsection*{Test:}
\subsubsection*{Pass Criteria:}


\subsubsection*{Functional Requirement 7}
\textit{EBI-bus for FPGA.}
\subsubsection*{Test:}
\subsubsection*{Pass Criteria:}


\subsubsection*{Functional Requirement 8}
\textit{Ring buffer}
\subsubsection*{Test:}
Create a program that needs a ring buffer. 
Simulate the program in ModelSim on a pipeline to verify that the ring buffer is working correctly.
Test the program on the FPGA and verify that the ring buffer works as in the simulation.
\subsubsection*{Pass Criteria:}
The FPGA outputs correct results and utilizes 


\subsection{Medium}
\subsubsection*{Functional Requirement 9}
\textit{Hardware-accelerate Fast Fourier Transform on the FPGA.}
\subsubsection*{Test:}
\subsubsection*{Pass Criteria:}

\subsubsection*{Functional Requirement 10}
\textit{SD card interface (SPI mode), in addition to serial port}
\subsubsection*{Test:}
\subsubsection*{Pass Criteria:}

\subsubsection*{Functional Requirement 11}
\textit{There should be buttons and LEDs (for status and control).}
\subsubsection*{Test:}
\subsubsection*{Pass Criteria:}


\subsection{Low}
\subsubsection*{Functional Requirement 12}
\textit{Take both digital and analog signals as input}
\subsubsection*{Test:}
\subsubsection*{Pass Criteria:}

\subsubsection*{Functional Requirement 13}
\textit{USB interface}
\subsubsection*{Test:}
\subsubsection*{Pass Criteria:}

