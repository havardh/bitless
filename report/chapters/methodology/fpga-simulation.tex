\section{FPGA simulation testing}\label{section:fpga-simulation-testing}

\todo[inline]{Simluation description (How simulation testing is performed).}

\subsection{ALU testing}

The testbench for the ALU is very straight forward. It tests every ALU operation
with two or three different sets of test values and then looks at the output.
The output consists of a result value and flags that represent how the ALU
operation executed. The input consists of four values; A, B, C, OPERATION, as
well as an enable signal. To test a simple integer operation you set OPERATION,
A and B to the desired values, for example ADD, 2 and 3, respectively. An assert
will then check that the result signal equals 5 after a small waiting period. C
and the enable signal are only used for certain floating-point operations.

\subsubsection{ALU test 1}

For all the integer operations, two to three sets of values were tested, ranging
from simple operations, like the above example, to the more complex operations.
Additionally, there were also a few tests to validate the values of the status flags. For the floating-point
operations, a slightly different testing methodology was used. The conversions
between integer and floating point was tested first. For these tests the
floating point values were calculated by hand and then compared to the result.
These operations were then used to simplify the rest of the floating point
testing. Values were converted to floating point, added, subtracted, or
multiplied in the fpu, and then converted back before they were tested with an
assert.

The abovementioned C value with accompanying enable signal are used by the
operation ``multiply and accumulate''. This operation needs a constant value
previously loaded from the constant memory into the ALU, in addition to the
traditional A and B values. The ALU stores the constant memory value as the C
value internally. When the operation executes, the enable signal activates the
writing of the internal register in the ALU. To test this operation A, B, C and
the enable signals were set to appropriate values and the result was checked if
equal ``A + B*C''.

\subsection{Core testing}
something, something

\subsubsection{Core test 1}

\subsubsection{Core test 2}

\subsection{Toplevel testing}
something, something

\subsubsection{Toplevel test 1}

\subsubsection{Toplevel test 2}
