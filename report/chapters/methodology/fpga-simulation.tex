\section{System Functionality Testing}

This section elaborates on some of the functional requirements shown in Section
\ref{intro:requirements}, and how we planned to test each of them. Not all the
requirements are listed here since many of them overlap each other. All of the following tests are black-box tests, and do not test any details of the system.

\begin{figure}[H]
    \label{test:func-test1}
    \begin{description}
        \item[Functional requirement 1] \hfill \\
            \textit{The audio pipelines must receive and execute different data and instruction streams.} \\
            \textbf{Test:} Initialize the audio processor with at least two pipelines
            containing one core each. Program the two cores with two different programs and feed
            each pipeline with different data.\\
            \textbf{Expected result:} The cores in each pipeline are able to read data from the input buffer,
            perform operations on the data and write the result to the pipeline's output buffers.
    \end{description}
\end{figure}

\begin{figure}[H]
    \label{test:func-test3}
    \begin{description}
        \item[Functional requirement 3] \hfill \\
            \textit{The audio processor must consist of at least two audio channels.} \\
            \textbf{Test:} Initialize the audio processor with two pipelines, where each of the
            pipelines perform calculations on an audio channel. Use a stereo wav-file to send
            the left channel through the first pipeline and the right channel through the
            second one. \\
            \textbf{Expected result:} The data received from the the output buffer is equal to the
            data sent to the input buffer.
    \end{description}
\end{figure}

\begin{figure}[H]
    \label{test:func-test6}
    \begin{description}
        \item[Functional requirement 6] \hfill \\
            \textit{The cores in the audio pipelines should share data with each other through a ring-buffer.} \\
            \textbf{Test:} Create a program that needs a ring buffer.
            Simulate the program in ModelSim on a pipeline to verify that
            the ring buffer is working correctly.
            Test the program on the FPGA and verify that the ring buffer
            works as in the simulation. \\
            \textbf{Expected result:} The FPGA outputs the same results as in the simulation, and utilizes at least one ring buffer.
    \end{description}
\end{figure}

\begin{figure}[H]
    \label{test:func-test7}
    \begin{description}
        \item[Functional requirement 7] \hfill \\
            \textit{It must be possible to observe power consumption over certain components.} \\
            \textbf{Test:} The PCB must be equipped with current sensors. Run programs
            on both the MCU and the FPGA, and measure the current passing through the sensors. \\
            \textbf{Expected result:} The measured results approximates the expected results.
    \end{description}
\end{figure}

\begin{figure}[H]
    \label{test:func-test8}
    \begin{description}
        \item[Functional requirement 8] \hfill \\
            \textit{The audio system should perform real time filtering on
            two distinct channels, with an audio quality of 8-bits and a frequency 44.1kHz.} \\
            \textbf{Test:} Setup the audio processor with two pipelines, each one with a core
            that filters the data passing through. Send the filtered audio to the DAC, with the
            right frequency. \\
            \textbf{Expected result:} The output sound has passable quality :-)
    \end{description}
\end{figure}

\begin{figure}[H]
    \label{test:func-test9}
    \begin{description}
        \item[Functional requirement 9] \hfill \\
            \textit{The microcontroller and the audio processor should communicate via ARMs
            External Bus Interface.} \\
            \textbf{Test:} Let \textit{ChaosM} respond to every response with all 16 bits set
            to high (0xffff) and read from the EBI-bank associated with the audio processor. \\
            \textbf{Expected result:} The read values match what is expected . This verifies that every data line is connected to both the FPGA and the MCU.
    \end{description}
\end{figure}

\begin{figure}[H]
    \label{test:func-test11}
    \begin{description}
        \item[Functional requirement 11] \hfill \\
            \textit{The MCU must have an SD card interface and a UART serial port.} \\
            \textbf{Test:} Copy a file to the SD card with the MCU, read the data on the SD card
            from an external device. Connect to the board with UART and write to an external device.\\
            \textbf{Expected result:} The file is copied correctly and the external device receives the
            transmitted data.
    \end{description}
\end{figure}

\begin{figure}[H]
    \label{test:func-test12}
    \begin{description}
        \item[Functional requirement 12] \hfill \\
            \textit{The board should have buttons and LEDs.} \\
            \textbf{Test:} Write a small program that reads the button status and sets
            the corresponding LED high. \\
            \textbf{Expected result:} All the LEDs light up when the corresponding buttons is depressed.
    \end{description}
\end{figure}

\begin{figure}[H]
    \label{test:func-test13}
    \begin{description}
        \item[Functional requirement 13] \hfill \\
            \textit{\textit{ChaosM} should be able to process both digital and analog signals.} \\
            \textbf{Test:} Run the audio processor with input data both from a Micro SD card, and
            with data from an ADC. \\
            \textbf{Expected result:} The resulting output from theThe PCB is equipped with both a Micro SD port and and ADC, and the
            MCU is able to fully control both of the peripherals.\\
    \end{description}
\end{figure}
