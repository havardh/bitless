% !TEX root = ../report.tex
\chapter{PCB}\label{chapter:pcb}

While the main objective of the project is to create a MIMD system, which essentially can be completed by FPGA design alone, everything depends on the PCB to complete the system. It is the final step in realizing the system as a whole, going from a theoretical dimension of design and ideas, to a physical one with components and electricity.

This chapter will attempt to explain in detail each step during the design process of the circuit board. The sections have been ordered as much as possible according to which stage in the process they occurred, with schematic drawings as the first step, choice of components as the second, followed by layout, and finally the test plan. While we have tried to divide these steps to our best ability, some of them are intertwined, like schematics and components, as you need of course to decide on what components your card will have before actually drawing them, unless you have a particular fondness for doing unnecessary work.

% !TEX root = ../../../report.tex

\subsection{Schematics}

The development tool used to design the PCB was Altium Designer 2013. Each sheet
contains components relevant to what is routed in it, consequently, large
components which consist of several block, are spread out across different
schematics. This was done because it was found to be more  intuitive to have as
many involved components as possible on the same sheet instead of connecting
loads of cables across sheets. The schematics themselves can be viewed in
Appendix~\ref{apx:schematics}.

From early on it was strongly advised to make things as easy as possible, so
many designs have been copied from other sources as it was safe to assume those
were already tested and confirmed to work. Designs like the opamp circuit for
the \emph{audio input}-port, and the power supply, have been copied from the
Silicon Labs development board and the energy efficiency group of 2012,
respectively. Component datasheets have also influenced much of the design,
since they provide a recipe for everything needed for the relevant component.

Another thing that has been prioritized is adding backup solutions and debugging
options in case something would fail due to PCB errors. \todo{add more about
being able to fall back on other protocols should one fail, and the mapping of
unused pins..}

% !TEX root = ../../../report.tex

\subsection{Components}

The PCB was produced using the same tools and technologies as is
commonly used by the industry, which influenced the choice of
components to be driven by a smaller is better mindset. The use
of BGA packages (for the MCU and FPGA) was already given from
the provided resources and the fact that Odd Rune Lykkebø had
experience baking these.

As much as possible was chosen as surface-mount components in
the 0805 size, because it was the recommended hand-soldering
size, and because this caused a reduction in the need for
through-hole drilling on the PCB. Not only did this lower the
production cost, but also provided more area to work on as it
was possible to place components on both sides, consequently
minimizing the board size.

Most components were picked from Farnell's inventory.
Supplementary components like capacitors and resistors were not
given a thorough evaluation as their specifications were listed
as requirements by the datasheets, but they had to follow the
rules of minimum size 0805 and be surface mounted.

\todo{add more components}

\begin{figure}[H]
    \centering
    \includegraphics[width=250px]{figures/sw/fpga.png}
    \caption{The enegy usage for DMA}
    \label{fig:fpga}
\end{figure}

% !TEX root = ../../../../report.tex

\subsubsection{SRAM}

To extend the memory of the MCU, an external memory module was added to the
board. During the evaluation of SRAM vs DRAM, SRAM came off as the chip of
choice due to several reasons. While DRAM is cheaper than SRAM, it was possible
to stay within the budget buying SRAM, and unlike DRAM, which is synchronous,
SRAM is asynchronous, which means there is no need to worry about the
propagation of signals and clock synchronization, making read/writes easier.
This is the main reason why SRAM was chosen.

As an added bonus, however, SRAM is more energy efficient since DRAM consumes
energy on regular intervals when it refreshes the memory values, while SRAM does
not refresh anything. This also makes SRAM faster.

Final choice fell upon a 16Mbit 1MHz chip that runs on 3.3V, surface mounted.

\missingfigure{On-board SRAM}

% !TEX root = ../../report.tex
\subsection{Power Supply}
\subsubsection{Overview}
Being on the energy efficiency group, maximum power utilization has been one of
our highest priorities. Therefore, we had to make sure that as little power as
possible went to waste during the transformation from whatever input we got to
the voltage levels used on-board. Before deciding on a power supply design we
looked at what the previous groups had done, particularly the energy efficiency
group of 2012, as they faced many of the same challenges as we have. Anything
before 2012 had more or less used the same power supply, which was
designed to transform any excess power into heat. This is a very simple and
effective design, but not very efficient in terms of power saving as it utilizes
only a fraction of the input power.

Fortunately for us, the energy efficiency group of 2012 did a solid job on their
power supply, using a switch-mode regulator that is able to output 3.3V from
any input voltage ranging from 4.75V up to 18V with an efficiency of 83-90\%, see
Appendix,\todo[inline]{add appendix sr10s3v3} so we copied most of their design
except that we changed the current sensors, which is explained in section~\ref{psu:current_sensors} - Current sensors.

\subsubsection{The design}
\paragraph{Switch-mode and linear regulators}
The design consists of a switch-mode regulator for 3.3V, a linear regulator for
1.2V and 1.8V, and two current sensors.\missingfigure{insert schematic drawing of
power supply, or place a reference to where it is located} The switch-mode gets an input voltage typically at
9V or 12V which is then transformed into 3.3V and kept stable with the help of
some capacitors. This in turn supplies the 3.3V input for the linear voltage
regulator that delivers both the 1.2V and the 1.8V output. Our original
intention was to disable the 1.8V domain entirely, running the entire board at
1.2V and 3.3V, but as we opted for a larger FPGA the FPGA flash had to be
powered with 1.8V. This setup delivers power at three different voltages and
does so rather efficiently. Very little power goes to waste in the switch-mode
regulator considering its efficiency as mentioned above, and the linear
regulator is \todo[inline]{Power Dissipation  = (Vin  - Vout) X Iout.... finish
later}

\paragraph{Current sensors} \label{psu:current_sensors}
The main issue with the original design was that the current sensors were not
sensitive enough for low voltages, which resulted in inaccurate measurements. To
remedy this we simply changed the current sensor. Finding the right one took
some research as most of the current sensors we could find were built for more
power hungry systems and were not very delicate in terms of supply voltage and
sensitivity. In general, measuring energy consumption at small scales \todo{insert something concrete here? ex: "such as 100uA"} can be challenging. In our case, we needed one that could run on 3.3V and
sense currents of the same range. Most current sensors need at least 5V supply,
or were not sensitive enough.\todo[inline]{verify or re-write. Add measurements
once tested on a working board.}

\paragraph{Power by USB} \label{psu:usb}
A feature we added from last year's group was the ability to power the board
from USB. The USB 1.x and 2.0 standard specification specifies that a constant
power draw of 500mA at 5.0V $\pm$ 5\% can be provided. \todo[inline]{Add
reference to the USB specification.} This should suffice for normal operation.
\todo[inline]{Test in practice. And also, what about changing to USB on the fly?}

% !TEX root = ../../report.tex

\subsubsection{USB-UART}

As a backup solution for communication with the MCU, other than the debug
port, a UART interface was added in form of a header that could be soldered
on to the board if needed. Should all other things fail, UART, beeing a simple
serial interface, would be a viable fallback solution. Both for debugging aswell
as piping sound in and out, although not in realtime. In order to improve 
usability of the serial interface a USB-UART bridge chip was added to the design
connected to a micro USB connector. 

While originally added as a backup solution for communication, the USB interface
was used for all its worth by the software group. And a script allowing for debugging
of the FPGA by facilitating communication between a PC and the FPGA using a specially
designed MCU program was added. \todo{Reference to the, hopefully existing, SW section
talking about the python script}

The USB connector, in addition to providing an interface to the MCU's UART
interface, also provides the second possible power source as outlined in 
section~\ref{psu:usb}.
% !TEX root = ../../report.tex
\subsection{Bus interface}
\subsubsection{Overview}

While not strictly a component the databus is still an important part of
our PCB design. The MCU natively supports two bus interfaces, I2C and EBI,
we opted for the latter. We did so because the I2C bus as a 
serial bus has a more limited bandwidth compared to the EBI bus which 
supports 16 bit words. The need for speed stems from the requirement of
streaming at least two audio streams live between the MCU and FPGA.
As an added bonus the EBI bus is compatible with SRAM chip interfaces 
which proved useful when including extra memory in our design.


In addition to the EBI bus we have a special control bus with a width
of 3 signals going between the MCU and FPGA. This bus is available for the
software and FPGA group to use as needed, for instance for interrupts or
other forms of synchronization and status signaling.

\subsection{Layout}
% !TEX root = ../../../report.tex

\subsubsection{Routing}

\todo[inline]{Runar, Terje?}
% !TEX root = ../../report.tex
\section{Layer Stack}

We had three primary voltage levels on the PCB -- 1.2V, 1.8V and 3.3V. To
minimize resistance in the power distribution system, we decided to dedicate a
layer on the PCB to be used as a shared power plane. Another plane was used for
common ground and covered the whole PCB (excluding non-grounded through-holes).
This setup with a shared ground and split power plane is common practice for
modern PCB designs.

As for the signal layers, the BGA packages we used only had a 0.8 mm pitch between
its feet. This forced us to have more signal layers than we originally set out
for, in order to fanout and escape the BGA packages. Using a QFP-like package
could probably have saved us some cost, but using BGA gave us a valuable
experience on modern PCB design.

We settled upon a total of 8 layers on the PCB. Compared to previous years projects
where mostly two signal and two power planes are used this is a significant increase.
The main reason we had to do this is that we, as previously mentioned, are opting for
BGA packages for our FPGA and microcontroller. Due to the ordering and pitch of the 
feets on the BGA package it is hard, if not impossible, to fan-out correctly on a 
two layer PCB. One added benefit from having 8 layers is that we were able to reduce
the overall size of the board. In the end we ended up with a board measuring only
120x150mm, which we figure is one of the smaller PCBs produced in this course.

% !TEX root = ../../../report.tex
\subsection{Footprints}

Each component has a 

% !TEX root = ../../report.tex
\section{Test plan}

In order to ease the process of assembly after receiving the final PCB from 
production, we set up a test plan detailing step by step what we are planning
to do. The main idea of the plan is catch potential errors in the PCB design
as early is possible so that if, and when, we run into a problem we can rule
out as many sources as possible.

\begin{enumerate}
    \item Manually test important solder points using a multimeter. This include connections between the power source and regulators. Aswell as the connection from ociliators and clock generators and their source pins on the microcontroller \
    and FPGA. After checking these points we can assume that the components will boot correctly.
    \item After soldering the power supply we will need to check the provided voltage levels. This is an important point as sensitive components might be damaged if these are wrong. We are able to do this safely even after the major components are soldered as we have added headers allowing us to disconnect the power planes from the regulators. \todo[inline]{Is this last statement correct?}
    \item  \todo[inline]{We should investigate JTag boundary scan as suggested by Gunnar}
    \item After soldering the major components aswell as JTAG interfaces we should attempt to flash test programs. These programs will be provided by the FPGA and Software groups and will test both the respective component, GPIO aswell as the interconnection between the two.
    \item After soldering the entire board we are going to test the more advanced user interfaces. This includes the UART over USB aswell as the audio input and output. Testing this will be done by a specially prepared program from the Software group.
    \item Finally after the FPGA and Software group have completed, or atleast mostly completed, their design we need to start measuring the overall power usage of the board. Using these measurement we will evaluate and possibly attempting to power our board using the 5v provided by the USB connection, as mentioned in the power supply section. \todo[inline]{ref} 
\end{enumerate}

% !TEX root = ../../report.tex

\section{Fixes}
\todo{Quick notes taken in the lab, needs a rewrite probably}

As the group had never before designed a PCB before, and since there are so
many things that can go wrong, obviously something was bound to not 
work as intended intially.

After receiving the PCB we quickly realised that the footprint for the 
linear regulator was miss-matched with the pinout of the component. 
After consulting our design project we realised that the footprint had
been fixed early on, but we had forgot to update the component within 
the schematic. The same turned out to have happend with the low frequency
crystal connected to the MCU.

Both the above problems where easly solved. For the crystal pin 1 and 4
was suppose to be used, but we had 1 and 2 in our design. As 2 and 3 are 
unconnected we where able to solve the problem by soldering a patch cable
between pin 4 and 2. \todo{Photo} 

The linear regulator was slightly more advanced as we had switched pin
1 and 5, and pin 2 and 4 in our design. The solution was soldering six 
short patch cables on the pad itself, and then mount the linear regulator
to those cables after ordering them correct. The resulted in the linear 
regulator being mounted about 1 cm above the card. \todo{photo, both physcal and a schematic maybe?}

In addition to the two footprints we had forgot to update we also had
an actual connection error in the SRAM footprint. One of the two chip
select inputs, CS1, was suppose to be grounded, but we had mistakenly connected
it to VCC33, resulting in the unit being stuck in off mode. To solve
this we slighly bent the chip select pin upwards, and connected it using
a patch cable to an other pin near by that was grounded. \todo{photo}