\section{Potential Improvements}

There have been several improvements we have considered for this project, and
below we will detail on the most prominent improvements that we were unable to
implement.

\subsection{ChaosM improvements}
\subsubsection{Further optimize the floating point operations}

If the floating point operations had been optimized further, the pipeline inside
of the core could have run faster, increaseing the speed of
\textit{ChaosM}. This consequently makes \textit{ChaosM} more energy efficient as the operations on the FPGA finish faster.

\subsubsection{Spend more effort on timing simulations}

The system timing simulations differ from behavourial simulations. Though all behavioral 
simulations worked as expected, their complementing timing simulations did 
not always simulate properly.
The key difference is that timing simulations simulate how the system will work
on the FPGA with signal propagations, in addtion to accounting for the time it
takes for signals to propagate through the logic circuits. This can reveal
locations in the VHDL design where a signal might become undefined, or have the
wrong value at the right time, where a behavourial simulation would not reveal
these problems. This is discussed more in Section \ref{conclusion:pitfalls}.

\subsection{I/O controller improvements}

There are no functional improvements to the I/O controller which were not
implemented into the project. Beyond functional improvements, there was a
wish to write a program which would show real-time graphs of the the audio
spectrums going into and out of the FPGA chip.

\subsection{PCB improvements}
reduce pcb cost by only using drill holes -

