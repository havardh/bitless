\section{Potential Improvements}

There have been several improvements we have considered for this project, and
below we will detail on the most prominent improvements that we were unable to
implement.

\subsection{ChaosM Improvements}

The processor core could be improved by predicting branch taken on every branch
instruction. This would reduce the number of no operation instructions to zero,
when the branch prediction is correct. A drawback would be increased complexity
in core, since it would require flushing three instructions when the prediction
is wrong. Since looping is a part of critical in audio processing programs;
branch prediction would be a vast improvement to the speed of the programs.
\subsubsection{Further optimize the floating point operations}

If the floating point operations had been optimized further, the pipeline inside
of the core could have run faster, increasing the speed of \textit{ChaosM}. This
consequently makes \textit{ChaosM} more energy efficient as the operations on
the FPGA finish faster.

\subsubsection{Timing simulations}
The system timing simulations differ from behavioral simulations. Though all
behavioral simulations worked as expected, their complementing timing
simulations did not always simulate properly. The key difference is that timing
simulations simulate how the system will work on the FPGA with signal
propagations, in addition to accounting for the time it takes for signals to
propagate through the logic circuits. This can reveal locations in the VHDL
design where a signal might become undefined, or have the wrong value at the
right time, where a behavioral simulation would not reveal these problems.

\subsection{I/O Controller Improvements}
There are no functional improvements to the I/O controller which were not
implemented into the project. Beyond functional improvements, there was a
wish to write a program which would show real-time graphs of the the audio
spectrums going into and out of the FPGA chip.

\subsection{PCB improvements}
The PCB cost can be reduced by only using drill holes, as the DC plug uses
special holes which adds expenses in production. Other than that, a reset
button, or just a simple power switch, would have been nice. Also, as an added
safety, being able to disconnect the power supply from the power planes would
have been nice to avoid frying any components should something go wrong.

There is a lot of static noise on the various signal lines. This affects sound
quality of the audio output. The audio lines
should probably have been shielded in order to avoid interference. 
