\FloatBarrier
\section{\textit{ChaosM}}\label{chapter:fpga}

\textit{ChaosM} consists of two audio processing pipelines, each containing four
homogenous processor cores. The processor cores are Turing complete processing
units connected to constant memory, an input buffer, an output buffer, and
instruction memory. Each core exchanges data through the input/output buffers.
Data from the I/O controller is written into the input buffer of the first core,
and read back from the output buffer of the last core.

With this design, each pipeline of cores can run a set of filters, for instance
a fourier transform, a frequency filter, an inverse fourier transform, and an
effect on the audio data stream running through the audio pipelines of
\textit{ChaosM}.

The rest of this section details several aspects of \textit{ChaosM}; the audio
pipelines, how \textit{ChaosM} communicates with the rest of the system, the
instruction set architecture, the different types of memory, the homogenous
processor core, the energy efficient implementations, and finally the assembler
used for programs that are run on \textit{ChaosM}.

\FloatBarrier
\subsection{Audio Pipelines}\label{subsec:audio_pipelines}

\missingfigure{Use the same diagram as below, but this picture needs
the same amount of cores, and the same programs, as the demo will run!}

\begin{figure}[H]
    \centering
    \includegraphics[height=150px]{figures/fpga/system_components_general_pipeline.png}
    \caption{Audio Pipeline Architecture}
    \label{fig:pipeline_architecture}
\end{figure}

\textit{ChaosM} consists of several audio processing pipelines, as illustrated in
figure \ref{fig:pipeline_architecture}. These contain several processing cores
separated by data buffers. Each audio processing pipeline processes one channel
of sound data.


\FloatBarrier
\section{Communication}\label{section:fpga-buses}

\subsection{The External Bus Interface}
The microcontroller communicates with the FPGA design using the External Bus
Interface (EBI) of the Giant Gecko microcontroller. The EBI is a parallel bus
with a separate data and address bus in addition to the chip select and read and
write enable signals, all active low\cite{efm_ebi}.

The communication between the MCU and the FPGA uses 23 address lines and 16
data lines. All transfers are initiated when the chip select signal goes low.
For write transfers, the data and address lines are then set up and the
write enable signal is asserted, see figure \ref{fig:ebi_write}. For reads,
the address is set up and the data line is put in high impedance mode before
the read enable signal is asserted, see figure \ref{fig:ebi_read}.

\begin{figure}[h]
	\centering
	\includegraphics[width=0.8\linewidth]{figures/fpga/ebi_write.png}
	\caption{EBI write transfer\cite[p.6]{efm_ebi}}
	\label{fig:ebi_write}
\end{figure}



\begin{figure}[h]
	\centering
	\includegraphics[width=0.8\linewidth]{figures/fpga/ebi_read.png}
	\caption{EBI read transfer\cite[p.7]{efm_ebi}}
	\label{fig:ebi_read}
\end{figure}



\FloatBarrier
\subsection{The Internal Bus}

The internal bus is used to transfer data to and from modules in the FPGA.
All transfers are initiated by the microcontroller on the EBI bus, and the
EBI controller facilitates communication between the EBI and the
internal bus.

\subsubsection{The EBI controller}
The EBI controller module is used to handle EBI transfers initiated by the
microcontroller. It consists of a simple state machine, illustrated in
figure \ref{fig:ebi_ctrl_fsm}. When an EBI transfer is executed, the
internal bus, described in is used to store or retrieve data from a module
in the FPGA.

The EBI controller keeps the EBI data lines in high impedance mode whenever
a transfer is not active, allowing the MCU to have control of the EBI bus.

\begin{figure}[h]
	\centering
	% TODO: Make the arrows in separate directions be separate arrows
	\begin{tikzpicture}[shorten >= 1pt,node distance=2cm,on grid,auto]
		\node[state,initial] (idle) {idle};
		\node[state] (read) [above right=of idle] {read};
		\node[state] (write) [below right=of idle] {write};
		\path[->]
			(idle) edge node {\small RE = 0} (read)
			       edge node [swap] {\small EW = 0} (write)
			(read) edge node {\small RE = 1} (idle)
			(write) edge node [swap] {\small WE = 1} (idle);
	\end{tikzpicture}
	\caption{EBI controller state machine}
	\label{fig:ebi_ctrl_fsm}
\end{figure}




\FloatBarrier
\subsubsection{Addressing}

Modules in the FPGA is addressed using a simple addressing scheme, where the
address is divided into several parts, as illustrated in figure \ref{fig:ebi_addresses}.

\begin{figure}[h]
	\centering
	\begin{bytefield}[bitwidth=0.04\linewidth]{23}
		\bitheader{0-22}\\
		\bitbox{1}{T} &
		\bitbox{2}{\tiny Pipeline} &
		\bitbox{4}{Device} &
		\bitbox{2}{\tiny Subdev} &
		\bitbox{14}{Address}
	\end{bytefield}
	\caption{FPGA address format}
	\label{fig:ebi_addresses}
\end{figure}




In an EBI address, the T bit is used to select the toplevel control register.
If the T bit is set, the rest of the address is ignored, and only the toplevel
control register is accessible.

If the T bit is not set, the pipeline field is used to select which pipeline
to address. In the pipeline, the device field is used to select
which pipeline module to address. Two device numbers have special meaning;
device 0 is the pipeline control register, while device 1 is the constant
memory. All device numbers starting at 2 accesses the cores in the pipeline.

The subdevice field is used to select between modules in each core. These
are listed in table \ref{tab:core_subdevices}.

\begin{table}[h]
	\centering
	\begin{tabular}{|l l|}
		\hline
		\textbf{Number} & \textbf{Description} \\
		\hline
		0 & Control register \\
		1 & Instruction memory \\
		2 & Input buffer \\
		3 & Ouput buffer \\
		\hline
	\end{tabular}

	\caption{Processor core subdevices}
	\label{tab:core_subdevices}
\end{table}


\FloatBarrier
\subsubsection{Read Transfers}

A read transfer is initiated when the read enable line from the microcontroller
goes low. The EBI controller sets up the internal address signals and asserts
the internal read enable signal. This causes the requested data to be available
in the next clock cycle. The EBI controller switches to read state, where it
remains until the chip select signal is deasserted.

\subsubsection{Write Transfers}

Write transfers are initiated the same way as read transfers. As the 
write-enable signal goes low, the destination address is latched into the 
internal address bus and the internal data lines are set to the value of the 
EBI data lines. The EBI controller enters write state and the internal
write enable signal is asserted. The idle state is reentered when the chip
select is deasserted.


% !TEX root = ../../report.tex
\subsection{Instruction Set Architecture}\label{section:fpga-isa}

The processor was designed in a top-down fashion, starting with
the instruction set. In order to support as many different filters
and effects as possible, the processor supports all normal arithmetic
operations. Due to the requirements of doing fourier transforms on the
processor, support for some floating point instructions was also included.

To make the decoding of instructions as easy as possible, instructions
were divided into three different instruction groups.

\subsubsection{Register-based Instructions}

The register-based instructions are instructions were both operands are
primarily registers. The group also includes a few instructions where
the second operand is an immediate value. The format of the
instructions are illustrated in figure \ref{fig:regbased_instrs_format}. The
implemented functions can be found in table \ref{tab:regbased_instrs}.

\todo{Fiks instruksjonstabellen}

\begin{figure}[h]
	\centering
	\begin{bytefield}[endianness=big,bitwidth=0.05\linewidth]{16}
		\bitheader{0-15}	\\
		\bitbox{2}{Group}	&
		\bitbox{2}{Funct}	&
		\bitbox{2}{Opt}		&
		\bitbox{5}{Reg A}	&
		\bitbox{5}{Reg B/Imm}
	\end{bytefield}

	\caption{Register-based instruction format}
	\label{fig:regbased_instrs_format}
\end{figure}

\begin{table}[H]
	\centering
	\begin{tabular}{|l l l l l|}
		\hline
		\textbf{Funct} & \textbf{Opt}  & \textbf{Mnemonic} & \textbf{Instruction} & \textbf{Operation} \\
	\hline
	\multicolumn{5}{|c|}{Group \texttt{0b00}} \\
	\hline
	\multirow{3}{*}{\texttt{0b00}}
		& \texttt{0b00} & \texttt{add \$ra, \$rb}  & Add registers & $\$ra \leftarrow \$ra + \$rb$ \\
		& \texttt{0b01} & \texttt{addi \$ra, imm} & Add immediate & $\$ra \leftarrow \$ra + imm$ \\
		& \texttt{0b10} & \texttt{fadd \$ra, \$rb} & Add registers (FP) & $\$ra \leftarrow \$ra + \$rb$ \\
	\multirow{4}{*}{\texttt{0b01}}
		& \texttt{0b00} & \texttt{sub \$ra, \$rb}  & Subtract registers & $\$ra \leftarrow \$ra - \$rb$ \\
		& \texttt{0b01} & \texttt{fsub \$ra, \$rb} & Subtract registers (FP) & $\$ra \leftarrow \$ra - \$rb$ \\
		& \texttt{0b10} & \texttt{cmp \$ra, \$rb}  & Compare & $cnd \leftarrow cnd(\$ra - \$rb)$ \\
		& \texttt{0b11} & - & compare fp / subtract imm? & - \\
	\multirow{4}{*}{\texttt{0b10}}
		& \texttt{0b00} & \texttt{mul \$ra, \$rb}  & Multiply registers & $\$ra \leftarrow \$ra * \$rb$ \\
		& \texttt{0b01} & \texttt{fmul \$ra, \$rb} & Multiply registers (FP) & $\$ra \leftarrow \$ra * \$rb$ \\
		& \texttt{0b10} & \texttt{fmla \$ra, \$rb} & Multiply-and-accumulate (FP) & $\$ra \leftarrow \$ra + \$rb * \$rc$ \\
		& \texttt{0b11} & \texttt{fmls \$ra, \$rb} & Multiply-and-subtract (FP) & $\$ra \leftarrow \$ra - \$rb * \$rc$ \\
	\multirow{4}{*}{\texttt{0b11}}
		& \texttt{0b0x} & \texttt{halt} & Stop processor &  \\
		& \texttt{0b10} & \texttt{fmla \$ra, \$rb} & Multiply-and-accumulate (FP) & $\$ra \leftarrow \$ra + \$rb * \$rd$ \\
		& \texttt{0b11} & \texttt{fmls \$ra, \$rb} & Multiply-and-subtract (FP) & $\$ra \leftarrow \$ra - \$rb * \$rd$ \\
	\hline
	\multicolumn{5}{|c|}{Group \texttt{0b01}} \\
	\hline
	\multirow{2}{*}{\texttt{0b00}}
		& \texttt{0b00} & \texttt{and \$ra, \$rb} & And & $\$ra \leftarrow \$ra \wedge \$rb$ \\
		& \texttt{0b01} & \texttt{nand \$ra, \$rb} & Nand & $\$ra \leftarrow \neg(\$ra \wedge \$rb)$ \\
	\multirow{3}{*}{\texttt{0b01}}
		& \texttt{0b00} & \texttt{or \$ra, \$rb} & Or & $\$ra \leftarrow \$ra \vee \$rb$ \\
		& \texttt{0b01} & \texttt{nor \$ra, \$rb} & Nor & $\$ra \leftarrow \neg(\$ra \vee \$rb)$\\
		& \texttt{0b10} & \texttt{xor \$ra, \$rb} & Xor & $\$ra \leftarrow \$ra \oplus \$rb$\\
	\multirow{4}{*}{\texttt{0b10}}
		& \texttt{0b00} & \texttt{mov \$ra, \$rb} & Move & $\$ra \leftarrow \$rb$\\
		& \texttt{0b01} & \texttt{mvn \$ra, \$rb} & Move negative & $\$ra \leftarrow \neg\$rb$ \\
		& \texttt{0b10} & \texttt{i2f \$ra, \$rb} & Typecast (Int to FP) & $\$ra \leftarrow fp(\$rb)$ \\
		& \texttt{0b11} & \texttt{f2i \$ra, \$rb} & Typecast (Fp to int) & $\$ra \leftarrow int(\$rb)$ \\
	\multirow{4}{*}{\texttt{0b11}}
		& \texttt{0b00} & \texttt{lda \$ra, [\$rb]} & Load from input & $\$ra \leftarrow [\$rb]$ \\
		& \texttt{0b01} & \texttt{ldb \$ra, [\$rb]} & Load from output & $\$ra \leftarrow [\$rb]$ \\
		& \texttt{0b10} & \texttt{ldc \$ra, [\$rb]} & Load from constant buffer & $\$rd\ and\ \$rc \leftarrow [\$rb]$ \\
		& \texttt{0b11} & \texttt{stb \$ra, [\$rb]} & Store to output & $[\$rb] \leftarrow \$ra$ \\
	\hline
	\end{tabular}
	\caption{Register-based instruction list}
	\label{tab:regbased_instrs}
\end{table}



\subsubsection{Load Immediate Instruction}
The load immediate instruction is used to load an immediate constant into a
register. The format of the instruction can be found in figure
\ref{fig:ldi_format}. The value is loaded into register \texttt{\$r1}.

\begin{figure}[h]
	\centering
	\begin{bytefield}[endianness=big,bitwidth=0.05\linewidth]{16}
		\bitheader{0-15} \\
		\bitbox{2}{Group} &
		\bitbox{14}{Imm}
	\end{bytefield}

	\caption{Load immediate format}
	\label{fig:ldi_format}
\end{figure}


\subsubsection{Branch Instruction}
The branch instruction checks condition flags and jumps accordingly. By
checking for various combinations of the condition flags, many different
conditions can be checked for. The format of the instruction is illustrated
in figure \ref{fig:new_branch_format}. List for the encoding of the flags field
is coming later.

\begin{figure}[h]
	\centering
	\begin{bytefield}[endianness=big,bitwidth=0.05\linewidth]{16}
		\bitheader{0-15} \\
		\bitbox{2}{Group} &
		\bitbox{4}{Flags} &
		\bitbox{11}{Target}
	\end{bytefield}

	\caption{Branch instruction format}
	\label{fig:new_branch_format}
\end{figure}



\paragraph{Special Registers}

Due to the limited space in instruction words, only two registers at most can be
specified in an instruction. Some instructions, such as the load immediate and
load constant instructions do not specify any registers, only an immediate
offset constant.

The list of defined special registers can be found in table \ref{tab:specregs}.
The register number for these registers have not yet been determined.

\begin{centering}[H]
	\centering
	\begin{tabular}{|l p{10.5cm}|}
		\hline
		\textbf{Register name} & \textbf{Register purpose} \\
		\hline
		\texttt{r0} & Zero register, hard coded to always contain 0 \\
		\texttt{r1} & Immediate register, contains the result of an \textsc{ldi} instruction\\
		\texttt{rc} & Constant register, provides operand to \textsc{fmla} and \textsc{fmls} instructions. Lies within the ALU. \\
		\texttt{rc} & Constant register, provides operand to \textsc{fmla} and \textsc{fmls} instructions. Lies within the ALU. \\
		\hline
	\end{tabular}
	caption{List of special registers}
	\label{tab:specregs}
\end{centering}




%Should contain: Ringbuffer, Switch buffer, input/output buffers, constant memory, inst memory, read and write lines to the fpga from the ARM. Originally the ARM only needed access to the first and the last data buffer but for debugging purposes the ARM was given access to all buffers.
\subsection{Memory types}\label{subsec:fpga-memory}

The internal block RAMs in the FPGA were used to implement the different kinds
of memory inside the processor. Even though this is limited by the amount of
memory available, it guarantees equal access time for all the different types
of memory. It also precludes having to use external RAM, which would not be as
fast, besides consuming more energy, making the project less energy-efficient.

\subsubsection{Instruction memory}
Each processor core has a separate instruction memory. This memory is filled
with instructions defining the program the processor core will run. Before the
FPGA processor cores get set to run, the the MCU writes the programs for each
core into these instruction memories.

\subsubsection{Audio Pipeline Buffers}
The audio pipeline buffers are used for inter-core communication. These memory
modules are shared between two processing cores, except for the outermost cores
in the pipeline, where they are the input and output for the audio pipeline. The
memory modules can operate as ringbuffers or switching buffers. While the audio
pipeline's input/output memory modules can only be operated as ringbuffers,
while the rest of the memory modules might be operated in either manner.

Since the BRAM modules in the fpga only feature two in/out ports, only the
first input and last output data memories are avaliable for the MCU to
write/read.

Independently of the buffer mode, the previous processor in the audio pipeline
has read and write access to the shared memory module, while the next core has
read-only access.

Memory modules communication data samples representing values in the frequency
domain are operated as switching buffers. While miscellaneous samples and
samples from the time domain stored in memory modules which are operated a
ringbuffer. These memory modules rotate the ringbuffer one slot per new sample.
Half of ringbuffer memory is readable by the next processor in the audio
pipeline, while the previous one can read and write to the whole memory.
Standard behaviour is to write only to the upper half, in order to prevent a
Write After Read (WAR) and similar hazards.

Data stored in the frequency domain uses every available memory address, and as
such the next processor in the audio pipeline has to gain access to all recent
values. For this reason, memory modules holding frequency values have a
switching mode, where the two processor cores switch base address at each
sample's arrival.

\subsubsection{Constant memory}
The constant memory is used to store constants needed for the SD-FT transforms
in each audio pipeline. While this memory is read-only, each audio pipeline
contains its own constant memory, and this memory can only be accessed by two
audio pipeline processor cores simultaneously. It is the pipeline control
register, which is configured by the MCU, which decides which cores can access
the constant memory.

The reason why only two cores are permitted access to the constant memory, is
that the internal block RAMs in the FPGA only have have two independent
read ports, permitting a maximum of two simultaneous reads. If more processors
were to have access to the constant memories an arbitration unit would be
needed, and the advantage of using the block RAM to ensure sincle-cycle access
time would be lost.

\FloatBarrier
\section{Processor Core}\label{section:fpga-processor-core}
\todo[inline]{How the processor is designed and why}

\subsection{Considerations}

The processor core was implemented with focus on being quick
in order to process audio in real-time, as well as being power
efficient to the degree permitted by the FPGA chip.

In order for the core to be as efficient as possible, it was
neccessary to make a pipelined processor design.
\todo{Describe the processor pipeline}

\subsection{Floating-point implementation design choices}
\todo{?}

\subsection{Memory Accesses}
\todo[inline]{How the core accesses memory.}


% !TEX root = ../../../report.tex

\subsection{Energy Efficiency}

\textit{ChaosM} is designed with a basic idea of energy efficiency. Since the
different cores in each pipeline perform different tasks, it is natural for them
to finish independently of each other. Therefore, \textit{ChaosM} is designed so
that it is possible to turn off the individual cores once they finish operating.
This functionality has a neglegible effect on the power consumption of an FPGA
and therefore doesn't affect energy efficiency on this particular system. If the
\textit{ChaosM} design were to be produced as an integrated circuit
however, this functionality could improve energy efficiency.

% !TEX root = ../../../report.tex

\subsection{Assembler}

Writing programs for a self-constructed ISA is cumbersome without utility
programs like compilers or assemblers. Thus, a simple assembler was written in
Haskell; the \textit{Hassembler}. It counts about 250 lines of code and uses the
monadic parser combinator library Parsec \cite{parsec}. The source code in its entirety is
released under a BSD3 licence available for download at
\url{https://github.com/terjr/hassembler}.

