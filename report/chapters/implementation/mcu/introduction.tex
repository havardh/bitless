The software implemented is run on an ARM Cortex-M3 supplied by Energy Micro.
This chapter describes the modules that are executed on said chip. The
responsibility of the software components are mainly to produce audio samples to
the FPGA, and to receive the samples the FPGA produces as a result of the sound
manipulation.

It implements interfaces for the different I/O devices which the system as a
whole use as sources and destinations for the data samples. These include the
ADC and DAC connected to the minijack plugs on the card, or the MicroSD card
connected to the SDCard slot. In addition, a serial port is available for
communication with a PC.

The MCU also handles general I/O such as buttons for controlling the behavior
of the program at runtime and LEDs for visual feedback.

Another responsibility  of the MCU is to program the FPGA cores before the sound
manipulation is initialized \ref{appendix:starting-fpga}. This is done when the
MCU configures which filters are to be used on the FPGA. \todo[inline]{Most of this is too specific.
It should be a ``light'' introduction; needs to be rewritten. E.g.
give an overview of terminology (CPU vs MCU), what role the MCU
plays in the system compared to the rest, and why the chapter
has been structured so and so. -JK}
