% !TEX root = ../../../report.tex

\subsection{Energy Efficient Software}

The software layer provides the best potential for energy savings. This can be done by turning parts of the system off, allowing
even a running system to achieve a level of energy saving. A well known technique is using \textit{interrupts}, in contrast
to busy waiting. In addition, {\it sleep modes} lets the system power down while still retaining some functionality.

The Giant Gecko provides 5 execution modes, where 4 of the modes correspond to different levels
of sleep. For this project, sleep modes EM0, EM1, and EM3 was utilized.


\begin{description}
	\item[Energy Mode 0] - Run Mode \hfill \\
		Fully operational with all features available, used when handling
interrupts and while reading from SDCard.
	\item[Energy Mode 1] - Sleep Mode \hfill \\
		CPU turned off, used when \textit{ChaosM} performs filtering, required
by EBI.
	\item[Energy Mode 3] - Stop Mode \hfill \\
		Deepest sleep mode without reset required, used when Bitless is idle.
\end{description}

If the reader is interested in additional information, it
can be found in the Reference Manual\cite{efm32gg}.
