\subsection{Energy Efficient Software}

The software layer provides the most potential for energy savings, since the best
way to save power is to turn off the system completely. When this is done is decided by the
algorithm the software runs. Even so, there exists some techniques which can ensure
a level of energy saving, even while the system is running. The most commonly
know technique is \textit{interrupts}, in contrast to busy waiting. {\it Sleep
modes} are used for turning of parts of the system while still enabling some
processing.

The Giant Gecko provides 5 execution modes, where 4 of the modes correspond to different levels
of sleep. For this project, sleep modes EM0, EM1, and EM3 was utilized.


\begin{description}
	\item[Energy Mode 0] - Run Mode \hfill \\
		Fully operational with all features available, used when handling
interrupts and while reading from SDCard.
	\item[Energy Mode 1] - Sleep Mode \hfill \\
		CPU turned off, used when \textit{ChaosM} performs filtering, required
by EBI.
	\item[Energy Mode 3] - Stop Mode \hfill \\
		Deepest sleep mode without reset required, used when Bitless is idle.
\end{description}
Refer to the user manual for additional information \cite{efm32gg}.
