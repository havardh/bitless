\subsection{Energy Efficient Software}

The software layer provides the most potential for energy savings, as the best
way to save power is to turn off the system completely. This is decided by the
algorithm the software runs. Still, there are some techniques which can ensure
a level of energy saving, even while the system is running. The most commonly
know technique is {\bf interrupts}, in contrast to busy waiting. {\bf Sleep
modes} are used for turning of parts of the system while still enabling some
processing. {\bf DMA}, Direct Memory Access, enables movement of data while the
CPU is sleeping.

The Giant Gecko provides 5 execution modes, where 4 modes are different levels
of sleep. For this project sleep modes EM0, EM1 and EM3 was utilized.


\begin{description}
	\item[Energy Mode 0] - Run Mode \hfill \\
		Fully operational with all features available, used when handling
interrupts and while reading from SDCard.
	\item[Energy Mode 1] - Sleep Mode \hfill \\
		CPU turned off, used when the system is performing filtering, required
by EBI.
	\item[Energy Mode 3] - Stop Mode \hfill \\
		Deepest sleep mode without reset required, used when Bitless is idle.
\end{description}
Refere to the user manual for addition information \cite{efm32gg}.
