%!TEX root = ../../../report.tex

\subsection{Memory Types}\label{subsec:fpga-memory}

All \textit{ChaosM} working memory was implemented in FPGA block RAM to reduce complexity
and improve energy efficiency by reducing data movement. External RAM could
provide a significantly larger working memory, but at the cost of a long access time.

\subsubsection{Instruction Memory}
Each processor core has a separate instruction memory. This memory is used to
store the program that is run by a processor core.

\subsubsection{Audio Pipeline Buffers}
The audio pipeline buffers are used for passing data through the audio pipeline.
Each core has an input buffer and an output buffer, with the latter also serving
as working memory for the core. While a core has write access to all of the 
output buffer, the suggested programming model is to only operate on half the
address space, to ensure that the receiving core can safely read the other half.
The buffers have two different modes of operation, switching mode and ring mode:

In ring mode the base pointers for both cores are incremented at each sample
clock tick. This is done to make addressing data samples by their age simple,
which is a useful support for transform \footnote{For the SDFT this mode implements the sliding window in hardware.} and effect algorithms.

In switching mode the base address pointers for the writing and reading core are
updated at every sample clock tick. For programs operating on data in the
frequency domain, it is important to make all values avaliable to the reading core.

A core supplies an offset from it's base address, the sum representing the
actual memory address. This lets the cores view the address map as static with
respect to sample age, or the address of a frequency value.

\subsubsection{Constant Memory}
The constant memory is used to store constants needed for computations, mainly
the Fourier coefficients needed for the SDFT. Each audio pipeline contains one
constant memory which is shared between a two select processors in the
pipeline. This limitation is due to the block RAM on the FPGA used, which only 
has two available read ports per module.

