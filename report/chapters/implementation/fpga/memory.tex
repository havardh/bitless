%Should contain: Ringbuffer, Switch buffer, input/output buffers, constant memory, inst memory, read and write lines to the fpga from the ARM. Originally the ARM only needed access to the first and the last data buffer but for debugging purposes the ARM was given access to all buffers.
\subsection{Memory types}\label{subsec:fpga-memory}

The internal block RAMs in the FPGA were used to implement the different kinds
of memory inside the processor. Even though this is limited by the amount of
memory available, it guarantees equal access time for all the different types
of memory. It also precludes having to use external RAM, which would not be as
fast, besides consuming more energy, making the project less energy-efficient.

\subsubsection{Instruction memory}

Each processor core has a separate instruction memory. This memory is filled
with instructions defining the program the processor core will run. Before the
FPGA processor cores get set to run, the the MCU writes the programs for each
core into these instruction memories.

\subsubsection{Audio Pipeline Buffers}
\todo[inline]{Describe the need for memory to behave as both ringbuffer/queues,
and as switching buffers. Discuss why this is a good idea because of our
intended functionality, and how we utilize the FPGAs resources to implement it
in the best possible way.}

\subsubsection{Constant memory}

The constant memory is used to store constants needed for the SD-FT transforms
in each audio pipeline. While this memory is read-only, each audio pipeline
contains its own constant memory, and this memory can only be accessed by two
audio pipeline processor cores simultaneously. It is the pipeline control
register, which is configured by the MCU, which decides which cores can access
the constant memory.

The reason why only two cores are permitted access to the constant memory, is
due to the internal block RAMs in the FPGA which only have have two independent
read ports, permitting a maximum of two simultaneous reads. If more processors
were to have access to the constant memories an arbitration unit would be
needed, and the advantage of using the block RAM to ensure sincle-cycle access
time would be lost.
