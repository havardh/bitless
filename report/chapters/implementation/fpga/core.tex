\FloatBarrier
\subsection{Processor Core}\label{subsec:fpga-processor-core}
\todo[inline]{How the processor is designed and why}

\subsubsection{Considerations}

The processor core was implemented with focus on being quick in order to process
audio in real-time, as well as being power efficient to the degree permitted by
the FPGA chip.

In order for the core to be as efficient as possible, a pipelined
processor design was implemented. The pipelined core design consists of the
following stages:

\todo[inline]{If someone has a better way to describe the processor core than
giving a quick overview of the implementations for each pipeline stage, please,
do edit.}

\begin{enumerate}
	\item Instruction Fetch \label{stage:if}
	\item Instruction Decode \label{stage:id}
	\item Execute \label{stage:ex}
	\item Memory \label{stage:mem}
	\item Writeback \label{stage:wb}
\end{enumerate}

\todo[inline]{Is the above list correct? Please someone re-view and edit it so
that it is.}

\subsubsection{Floating-point implementation design choices}

Some audio filters are performed in the time domain, others are performed in the
frequency domain. To convert the audio samples from one domain to the other, a
Fourier-transform (FT) or its inverse is usually employed. A discrete FT is
perfectly suited for the sample based nature of an audio processor. The FT we
chose to utilize for our project is the
Sliding-Discrete-Fourier-transform, detailed further in appendix section
\ref{appendix:sd-ft}.

With the goal of having a real-time sound processing system, the algorithmic
complexity of the SD-FT would make it one of the more algorithmically
time-complex operations the processor core would have to complete within the
time window between two sample inputs.

Realizing a SD-FT on a computing device requires floating point arithmetics.
Therefore, to lower the algorithmic complexity required for the floationg point
calculations the decision was made to lower the maximum size of the integers in
the audio samples to 8-bits. The consequence of this is that the maximum sound
frequency was reduced from the normal of 44 Kilo Hertz, down to ca. 11. Thereby
reducing the range of sounds the processor core could alter/manipulate.
