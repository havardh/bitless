\FloatBarrier
\subsection{Processor Core}\label{subsec:fpga-processor-core}
\todo[inline]{How the processor is designed and why}

\subsubsection{Design Considerations}

The processor core is designed to be able to perform audio processing in real-time, and well as being as power efficient as possible. 
Since the FPGAs power usage is close to constant, the best way to save power is to reduce the time the whole board is active. 
The best way for the core to contribute to this is to execute its programs as fast as possible. Thus, the processor core has been designed by the two following principles:


\begin{itemize}
	\item As high throughput as possible.
	\item Support the instructions needed to perform the audio processing.
\end{itemize}

\subsubsection{Implementation}

In order for the core to be as efficient as possible, a pipelined processor
design was implemented. The pipelined core design consists of the following
stages:

\begin{enumerate}
	\item Instruction Fetch \label{stage:if}
	\item Instruction Decode \label{stage:id}
	\item Memory \label{stage:mem}
	\item Execute \label{stage:ex}
	\item Write back \label{stage:wb}
\end{enumerate}

A change from the classic pipelined processor designs, is that the memory stage comes before the execution stage. 
This is because the load and store instructions does not need the ALU to calculate the memory address. 
Thus the processor is able to prevent \textit{all} data dependecies by forwarding data from the different stages.

For simplicity, the core does not have a hazard control unit implemented for
branching. Instead it relies on the programmer to add two no-operations after a branch.




\FloatBarrier
\subsection{ALU}\label{subsec:fpga-alu}

Which operations were implemented in the ALU was decided based on behalf of what
was neccessary for the Fourier-transform, sound processing, and on whether the
instruction set had room for the encoding of the operations.

Division was not implemented in hardware in order for all instructions to be
single-cycle. However, division can be done using floating point multiplication
or by a regular long division algorithm in software.

The ALU implementation receives two 16-bit registers as inputs, and outputs a
32-bit register. However, only the lower 16-bits are currently used. Previously,
the intention was for multiplication operations to output all 32-bits.

\subsubsection{Internal memory}

Since the instruction words can only address two registers at the time, the ALU
unit can store the values previously loaded from constant memory internally.
``Multiply and accumulate'' is the longest critical path, and which is used
during the SD-FT \ref{appendix:sd-ft}. This instruction needs the constant value
previously loaded into the ALU with the ``Load from constant memory''
instruction. Hence why the ALU can store a constant memory value internally.

