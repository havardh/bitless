\FloatBarrier
\subsection{Audio Pipelines}\label{sec:audio_pipelines}

\missingfigure{Overview of the pipeline structure, preferably use the same
picture as in assignment.tex line 38. But in this one, specify the filters/
effects so that they match what we're running in the demo. Instead of all of
them saying ``could be this or that''.}

The two aforementioned audio pipelines consist of a series of processors
connected by buffers. The buffers operate either as ringbuffers or as buffers
switching between two sets of data. Figure \todo{fix reference to above picture
}XX illustrates the layout of the pipelines with an example of what programs the
FPGA can run.

The idea behind splitting up the processor cores into these two audio pipelines
is that each audio pipeline can run each its own sound channel. With two
audio pipelines the project can run stereo sound through the FPGA. With more
resources on the FPGA, there would be nothing hindering the programming of
sound manipulation for any number of sound channels simultaneously.


