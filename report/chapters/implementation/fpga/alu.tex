\FloatBarrier
\subsection{ALU}\label{subsec:fpga-alu}

Which operations were implemented in the ALU was decided based on behalf of what
was neccessary for the Fourier-transform, sound processing, and on whether the
instruction set had room for the encoding of the operations.

Division was not implemented in hardware in order for all instructions to be
single-cycle. However, division can be done using floating point multiplication
or by a regular long division algorithm in software.

The ALU implementation receives two 16-bit registers as inputs, and outputs a
32-bit register. However, only the lower 16-bits are currently used. Previously,
the intention was for multiplication operations to output all 32-bits.

\subsubsection{Internal memory}

Since the instruction words can only address two registers at the time, the ALU
unit can store the values previously loaded from constant memory internally.
``Multiply and accumulate'' is the longest critical path, and which is used
during the SD-FT \ref{appendix:sd-ft}. This instruction needs the constant value
previously loaded into the ALU with the ``Load from constant memory''
instruction. Hence why the ALU can store a constant memory value internally.
