% !TEX root = ../../../report.tex
\FloatBarrier
\subsubsection{ALU}\label{subsec:fpga-alu}

The set of supported ALU operations have been decided based on those required by
the application field. Most standard arithmetic operations have been
implemented, with the exception of shifting and division. These were left out
to make room for floating point instructions, but can be emulated with some
overhead.

\paragraph{Floating-point Implementation Design Choices}
The decision to implement floating point operations in the alu stemmed from the
belief that fixed-point or integer valued Fourier Transforms would provide
insufficient accuracy. Since the Xilinx IP Core generator tool can generate the
various components needed, the gain from precision was considered to outweigh the
performance cost.

Ideas for further improvement upon the implementation was to run the floating
point operations at a higher clock frequency, with internal control logic to
perform complex multiplication in one cpu cycle. This option was deemed to
complex for this project, and instead single-cycle multiply-and-accumulate was
included. This allows a complex multiplication to run in three cycles, utilizing
a lifting structure, see \cite{oraintara}.

\paragraph{ALU constant registers}
A limitation of 16-bit instructions is that it permits only two registers to be
addressed at a time. The multiply and accumulate class of instructions require
three values, $A += B*C$, and as such one of them needs to be addressed
implicitly.

Since these instructions were implemented to provide support for
complex multiplication, a solution to this problem was to store the current Fourier
Coefficients in the ALU itself. They are loaded from constant memory using a
specific load instruction, and lets the different MAC operations access each of
the constants.
