\FloatBarrier
\subsubsection{ALU}\label{subsec:fpga-alu}

The set of supported ALU operations have been decided based on those required by
 the application field. Most standard arithmetic operations have been
 implemented, with the exception of shifting and division. These were left out
 to make room for floating point instructions, but can be emulated with some overhead.
%Which operations were implemented in the ALU was decided based on what
%was neccessary for the Fourier-transform, sound processing, and on whether the
%instruction set had room for the encoding of the operations.

%Division was not implemented in hardware in order for all instructions to be
%single-cycle. Instead, division can be done using floating point multiplication
%or by a regular long division algorithm in software.

\paragraph{Floating-point Implementation Design Choices}
The decision to implement floating point operations in the alu stemmed from the
belief that fixed-point or integer valued Fourier Transforms would provide
insufficient accuracy. Since the Xilinx IP Core generator tool can generate the
various components needed, the gain from precision was considered to outweigh the
performance cost.

Ideas for further improvement upon the implementation was to run the floating
point operations at a higher clock frequency, with internal control logic to
perform complex multiplication in one cpu cycle. This option was deemed to
complex for this project, and instead single-cycle multiply-and-accumulate was 
included. This allows a complex multiplication to run in three cycles, utilizing
 a lifting structure, see \cite{TODO: IntFFT}. 

%Some audio filters are performed in the time domain, others are performed in the
%frequency domain. To convert the audio samples from one domain to the other, a
%fourier transform (FT) and its inverse is usually employed. A discrete FT is
%perfectly suited for the sample based nature of an audio processor. The FT we
%chose to utilize for our project is the
%sliding discrete fourier transform (SDFT), detailed in appendix \ref{appendix:SDFT}.
%
%With the goal of having a real-time sound processing system, the algorithmic
%complexity of the SDFT would make it one of the more algorithmically
%time-complex operations a processor core would have to complete within the
%time window between two sample inputs.
%
%Realizing an SDFT on a computing device requires floating point arithmetics.
%Therefore, to lower the algorithmic complexity required for the floating point
%calculations, a decision was made to lower the maximum size of the integers in
%the audio samples to 8 bits. The consequence of this is that the maximum sound
%frequency was reduced from the standard value of 44~KHz, down to approximately 11~KHz.
%This reduces the amount of samples the processor core can process.

\paragraph{ALU constant registers}
A limitation of 16-bit instructions is that it permits only two registers to be
addressed at a time. The multiply and accumulate class of instructions require
three values, $A += B*C$, and as such one of them needs to be addressed
implicitly.\\
Since these instructions were implemented to provide support for
complex multiplication, a solution to this problem was to store the current Fourier
Coefficients in the ALU itself. They are loaded from constant memory using a
specific load instruction, and lets the different MAC operations access each of
the constants.

%Since the instruction words can only address two registers at the time, the ALU
%unit can store the values previously loaded from constant memory internally.
%``Multiply and accumulate'' is the longest critical path, and is used
%during the SD-FT \ref{appendix:SDFT}. This instruction needs the constant value
%previously loaded into the ALU with the ``Load from constant memory''
%instruction, which is why the ALU can store a constant memory value internally.
