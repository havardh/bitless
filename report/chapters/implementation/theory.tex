\section{Theory}

One of the initial goals was to be able to perform a Fourier Transform on the
audio stream and perform frequency domain filtering. A number of variations of
the DFT were considered and espescially the Fast Fourier Transform, Integer Fast
Fourier Transform and Sliding Discrete Fourier Transform algorithms were of
interest, see Appendix \ref{appendix:FT_in_DSP}.

The FFT was first considered, being the most well known and documented approach,
and had the advantage of being avaliable in the Xilinx ISE suite. Due to the
complexity of the algorithm, both the generated circuit's resource usage and the
algorithm run time were unsatisfactory.

An optimization of the FFT called the IntFFT \cite{oraintara}, was also
considered, having many desirable properties. Documentation of the algorithm was
deemed inadequate for implementation, since the overall complexity of the
algorithm remained the same.

The choice fell upon the SDFT\cite{jacobsen03}, being well suited for real-time
or streaming applications. A more in-depth comparison is given in Appendix
\ref{appendix:DFT}, but the main advantages was a linear running time for
streaming input and a simple implementation of the algorithm. The continued
design process therefore optimized with this algorithm in mind.
