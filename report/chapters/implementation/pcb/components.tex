% !TEX root = ../../../report.tex

\subsection{Components}

The PCB was produced using the same tools and technologies that are commonly
used by the industry, which influenced the choice of components to be driven by
a ``smaller is better'' mindset. The use of BGA packages (for the MCU and FPGA)
was already given from the provided resources and the fact that Odd Rune Lykkebø
had experience baking these.

For most components, \emph{surface-mount device} (SMD) versions were
preferred, because this caused a reduction in the need for through-hole drilling
on the PCB. For capacitors and resistors, which were all SMDs, 0805 was the chosen size because it
was the recommended minimum hand-soldering size. Not only did this lower the
production cost, but also provided more area to work on as it was possible to
place components on both sides, consequently minimizing the board size.

Most components were picked from Farnell's inventory. Supplementary components
like capacitors and resistors were not given a thorough evaluation since their
specifications were listed as requirements by the datasheets, but they had to
follow the rules of minimum size 0805 and be surface mounted.

\todo{add more components}

\begin{figure}[H]
    \centering
    \includegraphics[width=250px]{figures/sw/fpga.png}
    \caption{The enegy usage for DMA}
    \label{fig:fpga}
\end{figure}

% !TEX root = ../../../../report.tex

\subsubsection{SRAM}

To extend the memory of the MCU, an external memory module was added to the
board. During the evaluation of SRAM vs DRAM, SRAM came off as the chip of
choice due to several reasons. While DRAM is cheaper than SRAM, it was possible
to stay within the budget buying SRAM, and unlike DRAM, which is synchronous,
SRAM is asynchronous, which means there is no need to worry about the
propagation of signals and clock synchronization, making read/writes easier.
This is the main reason why SRAM was chosen.

As an added bonus, however, SRAM is more energy efficient since DRAM consumes
energy on regular intervals when it refreshes the memory values, while SRAM does
not refresh anything. This also makes SRAM faster.

Final choice fell upon a 16Mbit 1MHz chip that runs on 3.3V, surface mounted.

\missingfigure{On-board SRAM}

% !TEX root = ../../report.tex
\subsection{Power Supply}
\subsubsection{Overview}
Being on the energy efficiency group, maximum power utilization has been one of
our highest priorities. Therefore, we had to make sure that as little power as
possible went to waste during the transformation from whatever input we got to
the voltage levels used on-board. Before deciding on a power supply design we
looked at what the previous groups had done, particularly the energy efficiency
group of 2012, as they faced many of the same challenges as we have. Anything
before 2012 had more or less used the same power supply, which was
designed to transform any excess power into heat. This is a very simple and
effective design, but not very efficient in terms of power saving as it utilizes
only a fraction of the input power.

Fortunately for us, the energy efficiency group of 2012 did a solid job on their
power supply, using a switch-mode regulator that is able to output 3.3V from
any input voltage ranging from 4.75V up to 18V with an efficiency of 83-90\%, see
Appendix,\todo[inline]{add appendix sr10s3v3} so we copied most of their design
except that we changed the current sensors, which is explained in section~\ref{psu:current_sensors} - Current sensors.

\subsubsection{The design}
\paragraph{Switch-mode and linear regulators}
The design consists of a switch-mode regulator for 3.3V, a linear regulator for
1.2V and 1.8V, and two current sensors.\missingfigure{insert schematic drawing of
power supply, or place a reference to where it is located} The switch-mode gets an input voltage typically at
9V or 12V which is then transformed into 3.3V and kept stable with the help of
some capacitors. This in turn supplies the 3.3V input for the linear voltage
regulator that delivers both the 1.2V and the 1.8V output. Our original
intention was to disable the 1.8V domain entirely, running the entire board at
1.2V and 3.3V, but as we opted for a larger FPGA the FPGA flash had to be
powered with 1.8V. This setup delivers power at three different voltages and
does so rather efficiently. Very little power goes to waste in the switch-mode
regulator considering its efficiency as mentioned above, and the linear
regulator is \todo[inline]{Power Dissipation  = (Vin  - Vout) X Iout.... finish
later}

\paragraph{Current sensors} \label{psu:current_sensors}
The main issue with the original design was that the current sensors were not
sensitive enough for low voltages, which resulted in inaccurate measurements. To
remedy this we simply changed the current sensor. Finding the right one took
some research as most of the current sensors we could find were built for more
power hungry systems and were not very delicate in terms of supply voltage and
sensitivity. In general, measuring energy consumption at small scales \todo{insert something concrete here? ex: "such as 100uA"} can be challenging. In our case, we needed one that could run on 3.3V and
sense currents of the same range. Most current sensors need at least 5V supply,
or were not sensitive enough.\todo[inline]{verify or re-write. Add measurements
once tested on a working board.}

\paragraph{Power by USB} \label{psu:usb}
A feature we added from last year's group was the ability to power the board
from USB. The USB 1.x and 2.0 standard specification specifies that a constant
power draw of 500mA at 5.0V $\pm$ 5\% can be provided. \todo[inline]{Add
reference to the USB specification.} This should suffice for normal operation.
\todo[inline]{Test in practice. And also, what about changing to USB on the fly?}

% !TEX root = ../../report.tex

\subsubsection{USB-UART}

As a backup solution for communication with the MCU, other than the debug
port, a UART interface was added in form of a header that could be soldered
on to the board if needed. Should all other things fail, UART, beeing a simple
serial interface, would be a viable fallback solution. Both for debugging aswell
as piping sound in and out, although not in realtime. In order to improve 
usability of the serial interface a USB-UART bridge chip was added to the design
connected to a micro USB connector. 

While originally added as a backup solution for communication, the USB interface
was used for all its worth by the software group. And a script allowing for debugging
of the FPGA by facilitating communication between a PC and the FPGA using a specially
designed MCU program was added. \todo{Reference to the, hopefully existing, SW section
talking about the python script}

The USB connector, in addition to providing an interface to the MCU's UART
interface, also provides the second possible power source as outlined in 
section~\ref{psu:usb}.
% !TEX root = ../../report.tex
\subsection{Bus interface}
\subsubsection{Overview}

While not strictly a component the databus is still an important part of
our PCB design. The MCU natively supports two bus interfaces, I2C and EBI,
we opted for the latter. We did so because the I2C bus as a 
serial bus has a more limited bandwidth compared to the EBI bus which 
supports 16 bit words. The need for speed stems from the requirement of
streaming at least two audio streams live between the MCU and FPGA.
As an added bonus the EBI bus is compatible with SRAM chip interfaces 
which proved useful when including extra memory in our design.


In addition to the EBI bus we have a special control bus with a width
of 3 signals going between the MCU and FPGA. This bus is available for the
software and FPGA group to use as needed, for instance for interrupts or
other forms of synchronization and status signaling.
