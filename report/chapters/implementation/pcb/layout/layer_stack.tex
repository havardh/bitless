% !TEX root = ../../report.tex
\subsubsection{Layer Stack}

We had three primary voltage levels on the PCB -- 1.2V, 1.8V and 3.3V. To
minimize resistance in the power distribution system, we decided to dedicate a
layer on the PCB to be used as a shared power plane. Another plane was used for
common ground and covered the whole PCB (excluding non-grounded through-holes).
This setup with a shared ground and split power plane is common practice for
modern PCB designs.

As for the signal layers, the BGA packages we used only had a 0.8 mm pitch between
its feet. This forced us to have more signal layers than we originally set out
for, in order to fanout and escape the BGA packages. Using a QFP-like package
could probably have saved us some cost, but using BGA gave us a valuable
experience on modern PCB design.

We settled upon a total of 8 layers on the PCB. Compared to previous years projects
where mostly two signal and two power planes are used this is a significant increase.
The main reason we had to do this is that we, as previously mentioned, are opting for
BGA packages for our FPGA and microcontroller. Due to the ordering and pitch of the
feets on the BGA package it is hard, if not impossible, to fan-out correctly on a
two layer PCB. One added benefit from having 8 layers is that we were able to reduce
the overall size of the board. In the end we ended up with a board measuring only
120x150mm, which we figure is one of the smaller PCBs produced in this course.
