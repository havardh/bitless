% !TEX root = ../../../../report.tex
\subsubsection{Layer Stack}

There are three primary voltage levels on the PCB -- 1.2V, 1.8V and 3.3V. To
minimize resistance in the power distribution system, one layer on the PCB was dedicated
to be used as a shared power plane. Another plane was used for
common ground and covered the entire PCB, excluding non-grounded through-holes.
This setup with a shared ground and split power plane is common practice for
modern PCB designs.

As for the signal layers, the utilized BGA packages had only a 0.8 mm pitch between
the feet. This made it necessary to have more signal layers than originally intended,
in order to fanout and escape the BGA packages. Using something like a Quad Flat Package (QFP)
could probably have saved some cost, but using BGA gave a valuable
experience on modern PCB design.

The final count settled upon a total of 8 layers. Compared with previous projects,
where mostly two signal and two power planes are used, this is a significant increase.
The main reason for this, as previously mentioned, is due to the
BGA packages for the FPGA and MCU. The ordering and pitch of the
feet on the BGA package makes it hard, if not impossible, to fan-out correctly on a
two layer PCB.

The main benefits from using BGA and having 8 layers is the significant size reduction it enables, and that they are baked, which completely removes the manual soldering process. The board ended up measuring only
120x150mm, which is one of the smaller boards produced in the course.
