% !TEX root = ../../../../report.tex

\subsubsection{I/O}

I/O on the board is realized through micro-USB, microSD, buttons, LEDs, as
well as the audio circuits mentioned above, and a UART interface. The micro-USB was chosen for the
size, as it offers all the same features as the alternative USB and mini-USB ports. The same goes for the
microSD reader. LEDs were picked from stock SMD components at Farnell, and the
buttons were found on the lab.

As a backup solution for communication with the MCU other than the debug port,
the UART interface was added in form of a header. Should all other things fail,
UART, being a simple serial interface, would be a viable fallback solution, both
for debugging as well as piping sound in and out, although not in realtime. In
order to improve usability of the serial interface, a USB-UART bridge chip was
added to the design, connected to a micro USB connector.

The USB connector, in addition to providing an interface to the MCU's UART
interface, also provides a second power source as outlined in
the previous section~\ref{psu:usb}.
