% !TEX root = ../../../../report.tex
\subsubsection{Bus interface}

While not strictly a component, the databus is still an important part of
the PCB design. The MCU natively supports two bus interfaces, I2C and EBI,
of which the latter was used. This was done because I2C as a
serial bus has more limited bandwidth compared to the EBI, which
supports 16 bit words. The need for speed stems from the requirement of
streaming at least two audio streams live between the MCU and FPGA.
As an added bonus, the EBI bus is compatible with SRAM chip interfaces
which proved useful when including extra memory in the design.

In addition to the EBI bus there is a special control bus with a width
of 3 signals going between the MCU and FPGA. This bus is available for the
software and FPGA group to use as needed, for instance for interrupts or
other forms of synchronization and status signaling.