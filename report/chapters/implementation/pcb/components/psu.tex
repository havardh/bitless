% !TEX root = ../../../../report.tex

\subsubsection{Power Supply}

The power supply is a big part of the \todo{What design is the writer referring
to here?}design, consisting of several components. Being the energy efficiency
group, minimizing the maximum power utilization has been a high priority, making
it very important that as little power as possible goes to waste. When deciding
on a design, the work of previous groups was examined, particularly that of the
energy efficiency group of 2012. Anything before 2012 had to a bigger or lesser
extent used the same power supply, which was designed to transform any excess
power into heat. This is a simple and effective design, but not very efficient
in terms of power saving as it utilizes only a fraction of the input power.

Fortunately, the energy efficiency group of 2012 did a solid job on their power
supply, using a switch-mode regulator that is able to output 3.3V from any input
voltage ranging from 4.75V up to 18V with an efficiency of 83-90\% \todo{add
reference to sr10s3v3 datasheet}. This is very efficient, so most of the design
was copied except for a change in current sensors, which is explained in detail
below.

\paragraph{Voltage regulators}

The design is based on a switch-mode regulator that transforms DC input into
3.3V\todo{sr10s3v3 reference}, and a linear regulator\todo{linreg reference}
that transforms 3.3V into 1.2V and 1.8V, see Appendix~\ref{apx:schematics}
\todo{Keep this parenthesis for final report?}(File: power\_supply.SchDoc). The
switch-mode gets an input voltage typically at 9V or 12V which, as mentioned, is
transformed into 3.3V and kept stable by capacitors. This in turn supplies the
3.3V input for the linear voltage regulator that delivers both the 1.2V and the
1.8V output. Originally it was intended to disable the 1.8V domain entirely,
running the whole board on 1.2V and 3.3V, but as the FPGA was changed midway to
a slightly larger one, the FPGA flash also had to be changed and the new one
required 1.8V.

The setup delivers power at these three voltages, with high efficiency. Very
little power goes to waste in the switch-mode regulator, and while the linear
regulator transforms power into heat, transforming from 3.3V instead of, for
example 9V, is a large improvement.

\paragraph{Current sensors} \label{psu:current_sensors}

The main problem with the original design was that the current sensors were not
sensitive enough for low voltages, which resulted in inaccurate measurements. To
remedy this the current sensors were changed to more fitting ones. One has been
placed on the 3.3V power plane, while the other sits just before the linear
regulator. This way, it is possible to separate the 3.3V plane from the 1.2V and
1.8V, which enables analysis of MCU and FPGA power consumption in comparison
with the peripherals. In addition, the entire board consumption can be measured
at the PWR\_SRC pins located right after the DC input connector, see
Appendix~\ref{apx:schematics} \todo{Ref. last todo, keep this file mention
too?} (File: power\_supply.SchDoc).\todo{Refer to measurements section once it
is completed for more details.}

The output from the current sensors have each been connected to an ADC on the
MCU, which makes it possible to perform live measurements on-chip. However, by
performing measurements using the MCU, the results are skewed, as reading the
ADC increases power consumption. Therefore, measuring sockets have been added to
allow for reading the current sensor output on an external device, such as a
development board, making for more accurate readings.

\paragraph{Power by USB} \label{psu:usb}

An extra feature that was added was the ability to power the board from USB. The
USB 1.x and 2.0 standard specifies that a constant power draw of 500mA at 5.0V
$\pm$ 5\% can be provided\footnote{
http://www.usb.org/developers/docs/usb\_20\_070113.zip}. This should suffice for
normal operation. USB chargers will in most cases provide support for current
draws in the 1500mA to 2000mA range, so using one of these should
be able to supply sufficient power through the USB port.

Even when using the normal power socket as the primary power supply, parts of
the board will still be powered through USB. This includes the UART-USB bridge
and the related TX and RX LEDs. By powering these components with power from the
USB, all parts related to the USB connectivity will be completely powered down
when no USB cable is connected.
