% !TEX root = ../../../../report.tex

\subsubsection{SRAM}

To extend the memory of the MCU, an external memory module was added to the
board. There were two alternatives for external memory, \textit{Static Random Access
Memory} (SRAM), and \textit{Dynamic Random Access Memory} (DRAM). During the evaluation of SRAM
vs DRAM, SRAM came off as the chip of choice due to several reasons. While DRAM
is cheaper than SRAM, it was possible to stay within the budget buying SRAM.
DRAM is also synchronous while SRAM is asynchronous, which means
that for SRAM, propagation of signals and clock synchronization is not a
problem, making read/writes easier. This is the main reason why SRAM was chosen.

As an added bonus, however, SRAM is more energy efficient since DRAM consumes
energy on regular intervals when it refreshes the memory values, while SRAM does
not refresh anything. This also makes SRAM faster.

Final choice fell upon a 16Mbit 1MHz chip that runs on 3.3V, surface mounted.