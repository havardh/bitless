% !TEX root = ../../../report.tex

\subsection{Programming Interfaces}

There are two programmable devices in this system, the FPGA and the MCU.
The following describes how both components were programmed.

\subsubsection{FPGA Programming}
The FPGA can be programmed in numerous configurations, all of which have
distinct features and capabilities described in the
Spartan-6 FPGA Configuration User Guide\cite{fpga:config-user-guide}. To avoid ruining the programmability, a
decision was made to go for the simplest and less error-prone configuration; a
JTAG interface directly to the chip.

However, JTAG programming does not provide persistence -- the FPGA would have to
be reprogrammed every time it powered up. This was not an ideal solution, so a
FPGA configuration flash memory was added, from which the FPGA could read its
initial configuration at startup. Xilinx provides schematics to daisy-chain the
regular JTAG with a JTAG interface to the FPGA flash, making it easy to
implement. Using this setup both the FPGA and the flash will be programmed
simultaneously, and unlike the FPGA the flash will retain the program when
powered down. When powered up again the FPGA will look to the flash for its
initial program.

\subsubsection{MCU Programming}
Placing programming headers for the MCU on the PCB is a much simpler procedure.
The 20-pin ARM debug pinout is well documented in application notes and is
simple to set up.

In addition to programming the MCU, the debug header permits tracing of running
programs, as well as allowing the use of the energy profiler provided by Silicon Labs.
