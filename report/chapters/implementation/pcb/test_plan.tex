% !TEX root = ../../report.tex
\subsection{Test plan}

A rough step by step test plan was prepared by the PCB group in order to ease
the process of assembly of the final system, after receiving
the final PCB from production. The plan allows any errors, production or
otherwise, to be caught as early as possible and hence reduce the number of
wasted work hours due to a critcal error that could have been detected early
on.

\begin{enumerate}
    \item Manually test important solder points using a multimeter. This 
    includes connections between the power source and regulators. After 
    checking these points it is safe to assume that the connections within
    the power supply is as expected, and that short circuts due to design
    or production errors can be avoided.
    \item After soldering the power supply the provided voltage levels should
    be measured using the four measuring sockets available for each of the
    provided voltage levels and ground. This is an important step as sensitive
    components might be damaged if the provided levels are wrong. 
    \item After soldering the major components, as well as the JTAG and debug
    interfaces, specially prepared test program should be flashed. These will
    validate connections to the outside world, aswell as connections between 
    important components. These programs will be provided by the FPGA and 
    Software groups ahead of time.
    \item After soldering the entire board final integration tests where all
    provied user interfaces are to be done. This includes test of UART over 
    USB, SD card aswell as audio input and output interfaces. This phase of 
    the testing will be carried out by the software group, again using specially
    prepared test programs.
    \item Finally, after the FPGA and Software group have completed their work,
    or at least completed the design necessary to run any testing, the PCB group
    will begin measuring of the overall power usage of the board. These measurements
    will be used to evaluate the possibility of powering the board over USB, as 
    mentioned in section~\ref{psu:usb} - Power by USB.
\end{enumerate}
