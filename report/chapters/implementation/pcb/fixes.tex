% !TEX root = ../../report.tex

\subsection{Fixes}

As the group had never designed a PCB before, and since there
are so many things that can go wrong, obviously something was
bound to not work as intended.

After receiving the PCB and soldering up the power supply, smoke
appeared from the linear regulator upon plugging in the power
source. It turned out that the footprint was miss-matched with
the pinout of the component. The footprint had been fixed early
on in the process, but unfortunately the schematics had not been
updated. The same turned out to have happened with the low
frequency crystal connected to the MCU.

Both of the above problems where solved by extra wiring. For the
crystal, pin 1 and 4 was suppose to be used, but instead pin 1
and 2 were used in the design. As 2 and 3 are unconnected, a
patch cable was soldered between pin 4 and 2.
\missingfigure{picture with textual reference}

The linear regulator was slightly more advanced as pin 1 and 5,
and 2, 3 and 4, had been swapped. The solution was to solder six
short patch cables on the pads themselves, and then match the
feet of the linear regulator with the right cables. This
resulted in the linear regulator being mounted about 1 cm above
the card. \missingfigure{photo, both physical and a schematic maybe?
Remember textual reference.}

In addition, there was also an actual connection error in the
external RAM footprint. One of the two chip select inputs, CS1, was
suppose to be grounded, but had mistakenly been connected to
VCC33, resulting in the unit being stuck in off mode. To solve
this the chip select pin was bent slightly upwards, and then
connected with a patch cable to another pin nearby that was
grounded. \missingfigure{photo. Textual reference.}

Finally the data lines of the EBI bus was only connected to the
MCU and external RAM, due to a typo in the schematic. Again this
was solved by patch cables between the external RAM's data pins
and all available FPGA GPIO pins. As a result we where able to 
keep the bus as we originally inteded \todo{Making the assumption
that the third try will work..} but we lost all other IO 
opportunities on the FPGA. \missingfigure{photo of the hack}
