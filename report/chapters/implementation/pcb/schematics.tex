% !TEX root = ../../../report.tex

\subsection{Schematics}

The development tool used to design the PCB was Altium Designer 2013. Each sheet
contains components relevant to what is routed in it, consequently, large
components which consist of several block, are spread out across different
schematics. This was done because it was found to be more  intuitive to have as
many involved components as possible on the same sheet instead of connecting
loads of cables across sheets. The schematics themselves can be viewed in
Appendix~\ref{apx:schematics}.

From early on it was strongly advised to make things as easy as possible, so
many designs have been copied from other sources as it was safe to assume those
were already tested and confirmed to work. Designs like the opamp circuit for
the \emph{audio input}-port, and the power supply, have been copied from the
Silicon Labs development board and the energy efficiency group of 2012,
respectively. Component datasheets have also influenced much of the design,
since they provide a recipe for everything needed for the relevant component.

Another thing that has been prioritized is adding backup solutions and debugging
options in case something would fail due to PCB errors. The following was implemented for redundancy:

\begin{itemize}
 \item I/O - USB, microSD, UART, and minijacks
 \item Power - headers for the various power planes
 \item Memory - microSD and SRAM
 \item GPIO-pins - 12 for FPGA, 10 for MCU; used for any desired signals
\end{itemize}
