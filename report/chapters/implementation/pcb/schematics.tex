% !TEX root = ../../../report.tex

\subsection{Schematics}

The development tool used to design the PCB was Altium Designer 2013. Each sheet
contains components relevant to what is routed in it, consequently, large
components which consist of several block, are spread out across different
schematics. This was done because it was found to be more  intuitive to have as
many involved components as possible on the same sheet instead of connecting
loads of cables across sheets. The schematics themselves can be viewed in
Appendix~\ref{apx:schematics}.

From early on it was strongly advised to make things as easy as possible, so
many designs have been copied from other sources as it was safe to assume those
were already tested and confirmed to work. Designs like the opamp circuit for
the \emph{audio input}-port, and the power supply, have been copied from the
Silicon Labs development board and the energy efficiency group of 2012,
respectively. Component datasheets have also influenced much of the design,
since they provide a recipe for everything needed for the relevant component.

Another thing that has been prioritized is adding backup solutions and debugging
options in case something would fail due to PCB errors. \todo{add more about
being able to fall back on other protocols should one fail, and the mapping of
unused pins..}

% !TEX root = ../../../../report.tex

\subsubsection{Programming Interfaces}

\paragraph{FPGA Programming}
The FPGA can be programmed in numerous configurations, all of which have
distinct features and capabilities described in the \todo{insert reference?}
Spartan-6 FPGA Configuration User Guide. To avoid ruining the programmability, a
decision was made to go for the simplest and less error-prone configuration; a
JTAG interface directly to the chip.

However, JTAG programming does not provide persistence -- the FPGA would have to
be reprogrammed every time it powered up. This was not an ideal solution, so a
FPGA configuration flash memory was added, from which the FPGA could read its
initial configuration at startup. Xilinx provides schematics to daisy-chain the
regular JTAG with a JTAG interface to the FPGA flash, making it easy to
implement. Using this setup both the FPGA and the flash will be programmed
simultaneously, and unlike the FPGA the flash will retain the program when
powered down. When powered up again the FPGA will look to the flash for its
initial program.

\paragraph{MCU Programming}
Placing programming headers for the MCU on the PCB is a much simpler procedure.
The 20-pin ARM debug pinout is well documented in application notes and is
simple to set up.

In addition to programming the MCU, the debug header permits tracing of running
programs, as well as allowing the use of the energy profiler provided by Silicon Labs.

