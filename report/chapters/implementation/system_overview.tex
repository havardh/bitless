% !TEX root = ../../report.tex

\section{System Overview}

The system is built around an I/O control unit that organizes all data movement
on board. Analog audio signals enter the system through a minijack and gets
routed to the hardware accelerator by the I/O controller for audio manipulation.
At the same time, processed audio is routed back from the accelerator and to the
output minijack.

\subsection{System Components}
This section describes the physical components of the system.

\subsubsection{MCU}\label{intro:system-components-mcu}

The MCU handles all I/O, and communication with the \textit{ChaosM}.
Sound input is either gathered from a WAV file on the SD-card or as
a stream from the minijack on the board, and then passed on to the
\textit{ChaosM}. Integrated ADCs and DACs are used in conjunction with the
minijacks. In addition, it programs the \textit{ChaosM}. This provides a simple way to
change the \textit{ChaosM} behavior.

\subsubsection{FPGA}

The FPGA constitutes the audio processing environment. It is flashed with the
\textit{ChaosM} processor architecture, and processes input into output.

\subsubsection{Memory}

The MCU has an internal memory of 128KB. This is not sufficient when working
with audio files loaded from an SD-card, so the decision was made to add more
memory onto the board. In addition, if something were to go wrong with the
audio-in channel and SD-card, having extra memory can work as a third option for
storing input.

\subsubsection{I/O}

Adding input and output interfaces makes handling data easier, compared to going
through the debug interface to insert and fetch data from the chip. The
following components were added to the board.

\begin{enumerate}
	\item Micro-USB connection
	\item microSD memory card reader
	\item 3.5mm minijack connectors
	\item Buttons and LEDs
\end{enumerate}

\subsubsection{Bus interface}

The MCU natively supports two bus interfaces, \textit{Inter Integrated Circuit}
(IIC, more commonly referred to as I2C) and \textit{External Bus Interface} (EBI), of which
the latter was used. This was done because I2C as a serial bus has more limited
bandwidth compared to the EBI, which support 16 bit words. The need for speed
stems from the requirement of streaming at least two audio channels live between
the MCU and FPGA components. As an added bonus, the EBI bus is compatible with
SRAM chip interfaces which proved useful when including extra memory in the
design.

In addition to the EBI bus there is a special control bus with a width of 3
signals going between the MCU and FPGA. This bus is available for the software
and FPGA group to use as needed, for instance for interrupts or other forms of
synchronization and status signaling.
