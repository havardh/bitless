\section{Application and effect on Architecture}

\todo{This section is subject to change and movement and is mostly just notes}

The application chosen for this project is audio processing. The MCU unit will
provide audio samples to the FPGA which in turn will process the samples and
return the result. As the FPGA contains multiple cores multiple filers,
transforms and effects will be executed in parallell.

\subsection{Filters}

\begin{enumerate}
	\item FIR - Finite impulse response filter
	\item IRR - Infinite impulse response filter
\end{enumerate}

\begin{enumerate}
	\item Low pass filter
	\item High pass filter
	\item Band filter
\end{enumerate}

\subsection{Transformations}
Some audio filters are performed in the time domain, others are performed in the
frequency domain. \todo{illustration} To convert the audio samples from a domain
to the other a Fourier Transform or the inverse is usually employed. The
Discrete Fourier Transformed \todo{ref} is perfectly suited for the sample based
nature of a audio processor. This method has numerous optimized implementation,
most known is the Fast Fourier Transform\todo{ref}.

\subsubsection{Fast Fourier Transform}
The Fast Fourier Transform, FFT, takes a collection of N samples, $S$, as input
and returns the frequencies of this window in time, $FW$. On this frequency
window a frequency domain filter can be executed, which results in a filtered
window $FW_f$. This filtered window can then be tranformed back to a collection
of filtered samples $S_f$ by utilizing the inverse Fast Fourier transform IFFT.
The time complexity of the FFT is $O(n) = n*log(n)$, to get a smoother result
the transform is usually performed on overlapping windows of the sample stream.
\missingfigure{illustration of overlapping windows}

\subsubsection{Sliding Discrete Fourier Transform}
The Sliding Discrete Fourier Transform\todo{ref} takes one sample, $s_1$, a
frequency domain $FW_0$, and a historic sample $s_n$ as input along with some
constant coefficients.
It returnes the frequency domain window, $FW_1$ as altered by adding the input
sample, $s_1$ and removing the sample $s_n$. The fequency domain can be
initialized to all zeros and the same with the historic sample. The n value of
the historic sample correspond to the window size N in the FFT algorithmn. The
time complexity of the SDFT algorithm is $O(n) = n$. The inverse SDT takes the
frequency window and the same constant coefficients as input and return exactly
one sample $s_f$.

\subsubsection{Integer Discrete Fourier Transform}
Both the SDFT and FFT can be performed with both
\todo{WRONG FOR THE SD-FT}integer and floating point arithmetic. The advantage
of using integer is the speed of integer arithmetic compared to floating point.
The integer version performes a worse approximation than the floating point
version. \todo{Not so sure this is correct either. The reasoning I'm aware of is
that integer FFT (not SD-FT) requires much bigger integers than 8-bit to get
useful results, as well as more area on the FPGA chip.}

\begin{enumerate}
\item DFT - Discrete Fourier Transform
	\begin{itemize}
	  \item FFT - Fast Fourier Transform
		\item SDFT - Sliding Discrete Fourier Transform
		\item IDFT - Integer Discrete Fourier Transform
	\end{itemize}
\end{enumerate}

\subsection{Audio Effects}

\begin{enumerate}
	\item Reverb
	\item Echo
	\item Distortion
	\item Chorus
\end{enumerate}

\subsection{Effect on architecture}

\paragraph{Goal}
 - To make a pipelined MIMD with cores executing different transformes and
filters.
   E.g. Input $=>$ FFT $=>$ IIR $=>$ IFFT $=>$ Output

\paragraph{Buffering}
 - The cores in the pipeline should have a shared memory inbetween each core to
 communicate
   E.g. Mem $<=>$ Core $<=>$ Mem $<=>$ Core ... $<=>$ Mem

\begin{itemize}
	\item Insight A: FFT transformes a window of samples into a Frequency image
for this window. The window is created by N samples.
  \item Effect: Data will be transfered through the pipeline in big chuncks. One
new chunck will be transfered on each ADC cycle.
  \item Proposed solution: Each shared memory should be divided in two equaly
sized pieces, A and B. $Core_i$ uses $A_{i-1}$ as input and $B_{i+1}$ as
output on cycle $2j$ and $B_{i-1}$ as input and $A_{i+1}$ on cycle $2j+1$ for
$j = 0,1,2,3,4$
\end{itemize}


\begin{itemize}
	\item Insight $B_1$: If the a FFT is not applied a N bit sample will arive
at each ADC cycle.
  \item Insight $B_2$: A filter needs historic samples.
  \item Effect: Data will be transfered in small quantities, if the previously
scheme is used for samples at the start of the buffer historic data will not be
available.
  \item Proposed solution: Converge the buffer design to a circular buffer
design either at sample level or at some division of the buffer greater than 2.
\end{itemize}


