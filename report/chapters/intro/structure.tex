% !TEX root = ../../report.tex
\subsection{Report Structure}
\todo[inline]{Write how the report is structured. Wait with this till everything
is settled.}

\FloatBarrier
\subsubsection{System overview}
texttext lorem ipsum

\FloatBarrier
\subsubsection{FPGA chapter}
This chapter is structured in the following manner; First, in section
\ref{section:fpga-design} we describe the design of the processor core in more
detail, and explain the reasoning behind the general design principles we
followed. Second, in section \ref{section:sequential-pipeline} we describe the
aforementioned sequential pipeline in which the processor cores are implemented.
Third in section \ref{section:fpga-processor-core} we explain the reasoning and
design decisions behind the homogenous processor cores used in the pipeline.
Fourth, in section \ref{section:fpga-internal-memory} we continue with
describing the different types of memories that the processor as a whole with
all its cores utilize, from the instruction memory, to the data memory, and the
constant memory. Fifth, in section \ref{section:fpga-buses} we describe how the
processor this chapter so far has detailed communicates with the MCU and the
rest of the PCB through the EBI bus. Sixth, in section \ref{fpga-isa} we list
and describe how the assembly language the processor cores are designed for
works, as well as how it's meant to be utilized. Seventh, in section
\ref{section:fpga-testing} we conclude this chapter by explaining and showing
how the testing of the  \todo{Re-phrase this in a better way somehow?} VHDL/FPGA
implementation and methodology is discussed, including the subsequent results.

\FloatBarrier
\subsubsection{MCU chapter}

\FloatBarrier
\subsubsection{PCB chapter}

\FloatBarrier
\subsubsection{Discussion}
