% !TEX root = ../../report.tex
\section{Assignment}

Our interpretation of the assignment..

\section{Original Assignment Text}

\subsection{Task: Construct a Multiple Program, Multiple Data}
The performance increase available from harvesting Instruction Level Parallelism (ILP) from the serial 
instruction stream is limited because we have reached the maximum power consumption that can be 
handled without expensive cooling solutions [1]. Consequently, there is a significant interest in 
single-chip parallel processor solutions (e.g. [2,3]). The processor cores in commercial multi-core 
chips are conventional designs and therefore reasonably complex. In this work, your task is to design 
a multiprocessor,Multiple Instruction, Multiple Data streams (MIMD). A processor classified MIMD 
include a multiple-instruction stream organizations[5]. Such processors can executing different 
instructions, i.e. minimum 2 independent programs, on different (independent) data. 
 
The task also include that a suitable application is chosen to demonstrate the processor. 
Your processor will be implemented on an FPGA, and you are free to choose how to realize your  
MIMD  computer  architecture.  The system should  be shown to work with  a suitable  application. 
Studying the architecture of the Cell processor [7], or in general multi-core processors [6], can be a 
good starting point. And a final tip; Keep it simple, as simple as possible, but not simpler. 
Due to a large number of students this year, we will divide the work into two independent projects: 
a) Performance and b) Energy efficiency. The goal of group a) is to achieve maximum performance 
while group b) should try to balance performance and energy. The reports from both groups should 
include an evaluation of prototype performance and energy consumption. 

\subsection{Additional Requirements}
The unit must utilize an Energymicro micro controller and a Xilinx FPGA. The budget is 10.000 NOK, 
which must cover components and PCB production. The unit design must adhere to the limits set by 
the course staff at any given time. Deadlines are given in a separate time schedule.

\subsection{Evaluation}
The project is evaluated based on the project report and an oral presentation of the work as well as a 
prototype  demonstration.  One  grade  will  be  given  to  the  group  as  a  whole,  unless  there  are 
significant variations in the amount of effort put into the project.

\subsection{References}

\begin{enumerate}
	\item Olukotun and Hammond; The Future of Microprocessors; ACM Queue; 2005 
\item Bell et al.; TILE64 - Processor: A 64-Core SoC with Mesh Interconnect; ISSCC; 2008  
\item Kongetira et al.; Niagara: A 32-way Multithreaded Sparc Processor; IEEE MICRO; 2005 
\item Wikipedia; Goodyear MPP; http://en.wikipedia.org/wiki/Goodyear\_MPP 
\item M Flynn; Some Computer Organizations and Their Effectiveness; IEEE Transactions on Computers. Volume:C-21 ,  Issue: 9. 1972 
\item Wikipedia; Multi-core processor; http://en.wikipedia.org/wiki/Multi-core 
\item Wikipedia; Cell (microprocessor);http://en.wikipedia.org/wiki/Cell\_(microprocessor) 
\end{enumerate}
