\section*{Abstract}
As heat dissipation in processor designs increasingly becomes an issue,
multiprocessing has been identified as an effective source for performance
increase, without incurring the wrath of overheating. Utilizing two or more
cores in parallel to process data can be exploited to push performance beyond
that of a highly clocked single processor doing all the work. Setting up such
a system to take multiple instructions and multiple data (MIMD), the workflow
is not limited to data level parallelization like a SIMD system would be. With
this in mind, we settled on sound manipulation as the basis for our project,
figuring that if we build an array of processors, every column can correspond
to a particular filter, while every row is a different sound channel.
\newline

Bitless consists of an FPGA configured to realize said processor array, with
I/O implemented by audio jack channels and various other peripherals like micro
USB and microSD, which in turn are controlled by a microcontroller unit (MCU).
Everything has been placed on a custom designed PCB, and has been designed with
a high level of energy efficiency in mind.
\newline

The processor design has been kept as general as possible in order to make it
easy to implement, relying on instructions to produce different output for each
column of cores. Utilizing Fourier-transforms to enable many of the popular
sound-effect manipulations, the processor has the ability (and hardware
support), to perform Sliding-Discrete Fourier-transforms. This type of
Fourier-transform allows us to process and manipulate sound data in
\todo{Make sure we still want this ``real-time'' thingy in the abstract. Maybe
remove/re-phrase it.}real-time\footnote{Depending of course on the speed at
which the processor can compute.}.
\newline

\todo[inline]{mention something about the final performance of the system}
